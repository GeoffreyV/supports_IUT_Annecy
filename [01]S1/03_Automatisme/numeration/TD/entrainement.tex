\documentclass[noCustomPackages,QCM]{UPSTI_Document}
\usepackage{listings}
\usepackage{lstautogobble}
\usepackage{hyperref}
\usepackage{fontawesome}
\usepackage{multirow}
\usepackage{multicol}
\usepackage{currfile}
\usepackage{pdfpages}
\usepackage{placeins}

\usepackage{epstopdf}
\epstopdfsetup{outdir=./}
%\usepackage{pdftricks}




% \begin{psinputs}
%     \usepackage{pstricks}
%     \usepackage{pstricks-add}
%     \usepackage{pst-eucl}
%     \usepackage{color}
%     \usepackage{pstcol}
%     \usepackage{pst-plot}
%     \usepackage{pst-tree}
%     \usepackage{pst-eps}
%     \usepackage{multido}
%     \usepackage{pst-node}
%     \usepackage{pst-eps}
% \end{psinputs}

\definecolor{vert}{rgb}{0.0, 0.5, 0.0}

\newlength{\iconeHeight}
\setlength{\iconeHeight}{15pt}

\newcommand{\UPSTIvariante}{2}


\newlist{todolist}{itemize}{2}
\setlist[todolist]{label=$\square$}
\input{../../../preamble_semestre.tex}
\discipline{Automatisme}
\edef\CurrentFileDir{\currfiledir} 
\graphicspath{}

%\newcommand{\UPSTIsource}{Certaines images proviennent du site freepics.com}


\newcommand{\UPSTItitreEnTete}{Automatisme pour la robotique}
\newcommand{\UPSTItitre}{Configuration du réseau de la baie du robot}
\newcommand{\UPSTInumero}{2}

\documentVersion{E}
\newcommand{\UPSTInumeroVersion}{1}
\usepackage{csvsimple}

\newcommand{\sujet}{%
\exemplaire{1}{%
%\shorthandon{;} 3
%%% debut de l'en-tête des copies :

\noindent{\bf QCM  \hfill TEST}

\vspace*{.5cm}
\begin{minipage}{.4\linewidth}
  \centering\large\bf Histoire et géographie\\ Examen du 01/01/2008
\end{minipage}
\champnom{\fbox{\begin{minipage}{.5\linewidth}
\nom{} \prenom{} :

\vspace*{.5cm}\dotfill
\vspace*{1mm}
\end{minipage}}}

%%% fin de l'en-tête

\begin{center}
  \hrule\vspace{2mm}
  \bf\Large Géographie
  \vspace{1mm}\hrule
\end{center}

\restituegroupe{geographie}

\begin{center}
  \hrule\vspace{2mm}
  \bf\Large Histoire
  \vspace{2mm}\hrule
\end{center}

\restituegroupe{histoire}

\AMCassociation{\id}%association
%\AMCassociation[\nom-\prenom]{\id}4

}%fin commande \exemplaire
}%fin commande \newcommand

\begin{document}

%%% préparation des groupes

\setdefaultgroupmode{withreplacement}

\element{geographie}{
  \begin{question}{Paris}
    Dans quel continent se situe Paris~?
    \begin{reponses}
      \bonne{L'Europe}
      \mauvaise{L'Afrique}
      \mauvaise{L'Asie}
      \mauvaise{La planète Mars}
    \end{reponses}
  \end{question}
}

\element{geographie}{
  \begin{question}{Cameroun}
    Quelle est la capitale du Cameroun~?
    \begin{reponses}
      \bonne{Yaoundé}
      \mauvaise{Douala}
      \mauvaise{Abou-Dabi}
    \end{reponses}
  \end{question}
}

\element{histoire}{
  \begin{question}{Marignan}
    En quelle année a eu lieu la bataille de Marignan~?
    \begin{reponseshoriz}
      \bonne{1515}
      \mauvaise{1915}
      \mauvaise{1519}
    \end{reponseshoriz}
  \end{question}
}

\element{histoire}{
  \begin{questionmult}{Nantes}
    Que peut-on dire de l'Édit de Nantes~?
    \begin{reponses}
      \bonne{Il a été signé en 1598}
      \bonne{Il a été définitivement révoqué par Louis XIV}
      \mauvaise{Il a été signé par Henri II}
    \end{reponses}
  \end{questionmult}
}

%protection par mot de passe
%\AMCstudentslistfile{liste.csv}{id} 5

%%génération des sujets
\csvreader[head to column names]{liste.csv}{}{\sujet} 6
%\shorthandoff{;} 7
%\csvreader[head to column names,separator=semicolon]{liste.csv}{}{\sujet}8

\end{document}