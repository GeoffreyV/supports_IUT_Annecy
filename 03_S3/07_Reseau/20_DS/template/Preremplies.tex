\documentclass[QCM, noCustomPackages]{UPSTI_Document}
\usepackage{listings}
\usepackage{lstautogobble}
\usepackage{hyperref}
\usepackage{fontawesome}
\usepackage{multirow}
\usepackage{multicol}
\usepackage{currfile}
\usepackage{pdfpages}
\usepackage{placeins}

\usepackage{epstopdf}
\epstopdfsetup{outdir=./}
%\usepackage{pdftricks}




% \begin{psinputs}
%     \usepackage{pstricks}
%     \usepackage{pstricks-add}
%     \usepackage{pst-eucl}
%     \usepackage{color}
%     \usepackage{pstcol}
%     \usepackage{pst-plot}
%     \usepackage{pst-tree}
%     \usepackage{pst-eps}
%     \usepackage{multido}
%     \usepackage{pst-node}
%     \usepackage{pst-eps}
% \end{psinputs}

\definecolor{vert}{rgb}{0.0, 0.5, 0.0}

\newlength{\iconeHeight}
\setlength{\iconeHeight}{15pt}

\newcommand{\UPSTIvariante}{2}


\newlist{todolist}{itemize}{2}
\setlist[todolist]{label=$\square$}
\input{../../../preamble_Semestre.tex}
\discipline{Automatisme}
\edef\CurrentFileDir{\currfiledir} 
\graphicspath{}

%\newcommand{\UPSTIsource}{Certaines images proviennent du site freepics.com}


\newcommand{\UPSTItitreEnTete}{Automatisme pour la robotique}
\newcommand{\UPSTItitre}{Configuration du réseau de la baie du robot}
\newcommand{\UPSTInumero}{2}

\documentVersion{E}
\newcommand{\UPSTInumeroVersion}{0}
\newcommand{\UPSTImessage}{test} 

\newcommand{\sujet}{
\exemplaire{1}{%
\renewcommand{\UPSTImessage}{\champnom{\nom{} \prenom{} - Groupe \tp{}}} 
\renewcommand{\UPSTInumeroVersion}{\id{}}
\UPSTIbuildPage
\setcounter{figure}{0}
\setcounter{table}{0}
%%% debut de l'en-tête des copies : 

% \begin{center}
% \noindent{}\fbox{\vspace*{3mm}
%          \champnom{\Large\bf\nom{} \prenom{} - \id{}}\normalsize{}% 
%           \vspace*{3mm}
%       }
% \end{center}


\begin{center}\em
Durée : 7 minutes.

  Aucun document n'est autorisé.
  L'usage de la calculatrice est interdit.

  % Les questions faisant apparaître le symbole \multiSymbole{} peuvent
  % présenter zéro, une ou plusieurs bonnes réponses. Les autres ont
  % une unique bonne réponse.

  % Des points négatifs pourront être affectés à de \emph{très
  %   mauvaises} réponses.
\end{center}
\vspace{1ex}
%%% fin de l'en-tête

\restituegroupe{general}


\AMCassociation{\id}

%\AMCaddpagesto{3}
	  }
}


%%%%§§§§§§§§§§§§§§§§§§§§§§§§§§§§§§§§§

\begin{document}
%\UPSTIcompileVars
%%%Options

\AMCrandomseed{1237893}

\def\AMCformQuestion#1{{\sc Question #1 :}}

\setdefaultgroupmode{withoutreplacement}
%%% Fin Options
\element{general}{
  \begin{question}{ex01}
    Exercice 1
    \evaluationProfLignes{2}{1}
  \end{question}

  \begin{question}{ex02}
    Exercice 2
    \evaluationProfLignes{3}{2}
  \end{question}

  \begin{question}{ex03}
    Exercice 3
    \evaluationProfLignes{3}{2}
  \end{question}

  \begin{question}{ex04}
    Exercice 4
    \evaluationProfLignes{2}{2}
  \end{question}

  \begin{question}{ex05}
    Exercice 5
    \evaluationProfLignes{2}{2}
  \end{question}
  
  \begin{question}{ex06}
    Exercice 6
    \evaluationProfLignes{2}{2}
  \end{question}

  \begin{question}{ex07}
    Exercice 7
    \evaluationProfLignes{2}{2}
  \end{question}

  \begin{question}{CRTP}
    Compte-rendu de TP
    \evaluationProfLignes{4}{2}
  \end{question}

}
%%% groupes

%%%% fin des groupes

\csvreader[head to column names]{liste.csv}{}{\sujet}

\end{document}