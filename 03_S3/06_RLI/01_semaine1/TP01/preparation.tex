\subsection{Première communication}

\begin{UPSTIactivite}[2][Le bloc FbDALI\_Joblist][][][Préparation]
	\label{prepa:fbDALIJoblist}
	\UPSTIquestion{À l'aide de l'annexe  \ref{appendix:fbDALIJoblist}, page~\pageref{appendix:fbDALIJoblist}, expliquer le rôle de ce bloc fonction.}
	\vspace{1cm}
	\UPSTIquestion{Proposer une classe \textbf{CDaliJobList} pour éviter l'éparpillement des données.}\\
	\parbox[][3.5cm]{\textwidth}{}
	\UPSTIquestion{Dessiner le bloc fonction \textbf{FbDALI\_Joblist} permettant l'envoie de commandes au coupleur DALI. Utiliser la structure de données \textbf{daliJobList}, instance de \textbf{CDaliJobList}.\label{q:fbDALIJoblist}}
	\textit{On ne s'intéresse pas aux valeurs des variables.}
	\vspace{2cm}
\end{UPSTIactivite}

%Nous proposons d'utiliser un bloc fonction afin de commander l'éclairage d'un plafonnier. Ce fonctionnement est très simple :

\UPSTIboiteCentrale{Cahier des charges \textit{Plafonnier 1}}{
	\label{cdc:plaf1}
	\begin{itemize}
		\item Le plafonnier est commandé par le bouton blanc : chaque appui change son état (allumé ou éteint).
	\end{itemize}
}


\begin{UPSTIactivite}[2][Fonctionnement Télérupteur][][][Préparation]
	\label{prepa:telerupteur}%
	\UPSTIquestion{A l'aide de l'annexe \ref{appendix:fbDALILatchingRelay} page \pageref{appendix:fbDALILatchingRelay}, expliquer le rôle du bloc \textbf{FbDALI\_LatchingRelay}.}\vspace{1cm}
	\UPSTIquestion{Dessiner ce bloc ainsi que les valeurs à placer en Entrée et en Sortie pour réaliser le cahier des charges ci-dessus.\\\textit{Note : Certaines entrées peuvent être laissées vides.}\label{q:fbDALILatchingRelay}}\vspace{5.2cm}
\end{UPSTIactivite}


\subsection{Envoi de commandes avancées}
Nous nous intéressons dans cette section au bloc \textbf{FbDALI\_Master} (Annexe \ref{appendix:fbDALIMaster}, page~\pageref{appendix:fbDALIMaster}).

\begin{UPSTIactivite}[2][Classe Ecg pour FbDALI\_Master][][][Préparation]
	\label{act:fbDALIMaster}
	\UPSTIquestion{À l'aide de l'annexe \ref{appendix:fbDALIMaster}, page~\pageref{appendix:fbDALIMaster}, expliquer le fonctionnement de ce bloc.}
	\vspace{1.5cm}
	\UPSTIquestion{Proposer une structure de données \textbf{CEcg} qui permettra la gestion d'un ballast du réseau.}
	\vspace{5.5cm}
	%\UPSTIquestion{Ajouter une variable indiquant si une commande est en cours d'envoi.}
	\UPSTIquestion{Dessiner le bloc fonction \textbf{FbDALI\_Master} permettant d'envoyer une commande au ballast Rouge en utilisant la structure de données \textbf{ecgRouge}, instance de \textbf{CEcg}.}
	\vspace{5.5cm}
\end{UPSTIactivite}

\begin{UPSTIactivite}[2][La fonction FuDimmValue][][][Préparation]
	\UPSTIquestion{À l'aide de l'annexe \ref{appendix:fuDimmValue}, page~\pageref{appendix:fuDimmValue}, expliquer le fonctionnement de cette fonction.}
	\vspace{1.2cm}\\
	On veut éclairer le ballast rouge à 50\% de sa puissance maximale.
	\UPSTIquestion{Ecrire une ligne en ST qui utilise cette fonction pour préparer la commande à envoyer.}\\
	\\ecgRouge.m\_bCommandValue := \dots
\end{UPSTIactivite}

% \subsection{Mise en place de la stratégie d'éclairage}
% \UPSTIboiteCentrale{Cahier des charges 2}{
% 	\label{cdc:1}
% 	\begin{itemize}
% 		\item La lampe rouge est allumée à 100\% lorsque le bouton rouge est appuyé.
% 		\item La lampe rouge est éteinte lorsque le bouton rouge est relâché.
% 	\end{itemize}
% }

% Pour réaliser ce cahier des charges, on se propose d'agir directement sur l'intensité lumineuse (0\% ou 100\%) de la lampe par la commande standard \textbf{Direct control of lamp power}. On utilisera alors la fonction \textbf{FuDimmValue\_DALI} (Annexe \ref{appendix:fuDimmValue}, page~\pageref{appendix:fuDimmValue}).

% \begin{UPSTIactivite}[2][][][][Préparation]
% 	\UPSTIquestion{Écrire, en texte structuré, le programme permettant de réaliser le cahier des charges.}
% 	\textit{Nommer correctement les variables utilisées.}
% 	\vspace{9cm}
% \end{UPSTIactivite}




