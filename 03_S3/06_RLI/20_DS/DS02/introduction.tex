

%Un extrait du CCTP (Cahier des Clauses Techniques Particulières) est donné en annexe. Il présente les exigences du client en termes de gestion technique du bâtiment. Certaines questions y feront référence. 




\section{Gestion de l'éclairage d'un bâtiment}
\subsection{Etude de l'architecture}
On s'intéresse à la gestion d'éclairage d'un bâtiment de 5 étages.  

\UPSTIquestion{Quels sont les intérêts d'utiliser un bus pour la gestion de l'éclairage d'un bâtiment ?}

\UPSTIquestion{Le bus DALI utilise comme couche physique une paire de fils différentielle. Quels sont les avantages de ce type de câblage ?}

\UPSTIquestion{On recommande une puissance d’éclairage entre 6 et 12 W/m². Chaque étage ayant une surface de 6000 m² et chaque lampe faisant environ 120W, combien de luminaires seront nécessaires ?}
\UPSTIquestion{Combien de luminaires peut-on installer sur un même bus DALI ?} 

\UPSTIquestion{Un étage comprend environ 250 luminaires. Quelle solution proposez-vous ?}

Le bus DALI v2 permet d'y accoupler des capteurs multifonction : (luminosité, présence, réception télécommande) et des boutons poussoirs. Ces capteurs et boutons sont alimentés par le bus.

\UPSTIquestion{Donnez un schéma synoptique du réseau DALI d'un étage comprenant :}
\begin{itemize}
	

\item un automate WAGO avec une ou des bornes DALI,
\item le réseau électrique 230 Vac,
\item 5 branches de 50 ballasts DALI (ne pas dessiner les 250 ballasts!),
\item l'alimentation 18 Vdc DALI
\item 5 ensembles de 4 capteurs multifonctions

\end{itemize}

\subsection{Programmation}
Pour une salle de projection, on propose le cahier des charges suivant :

\UPSTIboiteCentrale{Cahier des charges : Salle de projection}{
    \begin{itemize}
        \item L'appui sur un bouton poussoir \textbf{ixBpLuminaire} permet d'allumer ou d'éteindre la salle. 
    \end{itemize}}

%\UPSTIquestion{A partir de l'annexe, expliquer le principe de fonctionnement du bloc \textit{FbDALI\_DimmEasy}.}

%\UPSTIinfo[bReferenceadress1 et bReferenceaddress2]{La variable \textbf{bReferenceaddress1} ne sera utilisée que dans le cas d'un adressage par groupe. Elle sert à spécifier un luminaire de référence dans le groupe. Ce luminaire pourra être interrogé pour connaitre la valeur de l'intensité lumineuse du groupe. \textbf{bReferenceaddress2} est donnée pour assurer une redondance.}

\UPSTIquestion{Écrire un programme, en langage CFC (Blocs fonctions), permettant de répondre au cahier des charges.}

