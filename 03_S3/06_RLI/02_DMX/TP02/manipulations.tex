\section{Partie pratique}

\resetNumActivite{}
\subsection{Prise en main}
%\begin{UPSTIactivite}[][Ouverture du projet]
    \UPSTIetape{Récupérer le projet DMX dans le dossier \textit{Squelettes} du module, le copier dans votre dossier personnel puis l'ouvrir dans une machine virtuelle.}
%\end{UPSTIactivite}

\UPSTIboiteCentrale{Cahier des charges : Une couleur de chaque côté}{
    On souhaite allumer la colonne de gauche (Canal 1) d'une couleur de votre choix, la colonne de droite (canal 7) d'une autre couleur. 
}

\begin{UPSTIactivite}[][Premier programme en langage CFC]
    Dans cette activité, pour simplifier et appréhender les interactions entre les blocs, nous nous affranchissons temporairement de la notion de classe et utiliserons uniquement des blocs fonctions.\\
    \UPSTIetape{Mettre en oeuvre le cahier des charges ci-dessus en langage CFC à l'aide des blocs fonctions \textit{FbDMX\_Master}, \textit{FbRGB\_ColourMixer}.}
    \UPSTIetape{Tester puis faire vérifier par l'enseignant.}
\end{UPSTIactivite}


\subsection{Programmation en ST}

\begin{UPSTIactivite}[][Sucession de couleurs]
    Dans cette activité, nous nous proposons d'implémenter le cahier des charges \textit{Sucession de couleurs} donné en page~\pageref{CDC:succession}.\\
    \UPSTIetape{Désactiver ou supprimer le programme précédent.}
    \UPSTIetape{Déclarer les classes \textit{CDMXMaster}, et \textit{CRGBCrossFadeSequence} dans l'onglet type de données.}
    \UPSTIetape{Mettre en œuvre le cahier des charges ci-dessus en langage ST dans le module \textbf{DmxPrg} en instanciant des objets des classes créées. On pensera à appeler le bloc fonction de chaque classe.}
    \UPSTIetape{Tester puis faire vérifier par l'enseignant.}
\end{UPSTIactivite}

\UPSTIboiteCentrale{Cahier des charges : Potentiomètre}{
    La couleur du bandeau de gauche doit représenter la valeur de la tension aux bornes du potentiomètre. Cette valeur est disponible grâce à la variable \textit{bValue} et est comprise entre 0 et 255.
}

\begin{UPSTIactivite}[][Déclaration des classes]
    \UPSTIetape{Déclarer une classe \textit{CRGB\_ColourPalette} pour stocker les couleurs de la palette.}
    \UPSTIetape{Lui ajouter une méthode de type \textit{FbRGB\_RecallColourPalette} pour sélectionner une couleur de la palette. }
    \UPSTIetape{Instancier un objet de cette classe, puis définir le tableau des couleurs (on utilisera un tableau identique à celui de l'activité précédente.)}
    \UPSTIetape{Pour chaque entrée \textit{xRecallColour\_x} du bloc fonction, écrire une condition fonction de la valeur du potentiomètre pour sélectionner la couleur correspondante. L'ordre devra être le même que pour la séquence de couleur.}
\end{UPSTIactivite}

\subsection{Pour aller plus loin}

\UPSTIboiteCentrale{Cahier des charges : Modification de la séquence}{
    Un fois une couleur affichée sur le bandeau de gauche, on souhaite pouvoir modifier cette couleur à l'aide des boutons poussoirs et du potentiomètre. On propose la démarche suivante : 

    \begin{enumerate}
        \item Lorsque le potentiomètre est tourné, la couleur varie comme pour le cahier des charges précédent. 
        \item Si on souhaite modifier cette couleur, on actionnera le bouton blanc. 
        \item La couleur en question (case du tableau) est alors sélectionnée. La couleur ne varie plus directement avec le potentiomètre, mais peut être modifiée des boutons rouge, vert et bleu avec le potentiomètre. 
        \begin{enumerate}
            \item Lorsque le bouton rouge est actionné, la couleur rouge varie avec le potentiomètre.
            \item Lorsque le bouton vert est actionné, la couleur verte varie avec le potentiomètre.
            \item Lorsque le bouton bleu est actionné, la couleur bleue varie avec le potentiomètre.
        \end{enumerate}
        \item Lorsque le bouton blanc est actionné à nouveau, la couleur est validée et la séquence reprend.
    \end{enumerate}
}

\begin{UPSTIactivite}[][Pour aller plus loin]
    \UPSTIetape{Proposer une démarche pour répondre au cahier des charges ci-dessus. La faire valider par l'enseignant}
    \vspace{10cm}
    \UPSTIetape{Mettre en œuvre cette démarche.}
\end{UPSTIactivite}
