\documentclass[TP, noCustomPackages]{UPSTI_Document}
\usepackage{listings}
\usepackage{lstautogobble}
\usepackage{hyperref}
\usepackage{fontawesome}
\usepackage{multirow}
\usepackage{multicol}
\usepackage{currfile}
\usepackage{pdfpages}
\usepackage{placeins}

\usepackage{epstopdf}
\epstopdfsetup{outdir=./}
%\usepackage{pdftricks}




% \begin{psinputs}
%     \usepackage{pstricks}
%     \usepackage{pstricks-add}
%     \usepackage{pst-eucl}
%     \usepackage{color}
%     \usepackage{pstcol}
%     \usepackage{pst-plot}
%     \usepackage{pst-tree}
%     \usepackage{pst-eps}
%     \usepackage{multido}
%     \usepackage{pst-node}
%     \usepackage{pst-eps}
% \end{psinputs}

\definecolor{vert}{rgb}{0.0, 0.5, 0.0}

\newlength{\iconeHeight}
\setlength{\iconeHeight}{15pt}

\newcommand{\UPSTIvariante}{2}


\newlist{todolist}{itemize}{2}
\setlist[todolist]{label=$\square$}
\input{../../../preamble_semestre.tex}
\discipline{Automatisme}
\edef\CurrentFileDir{\currfiledir} 
\graphicspath{}

%\newcommand{\UPSTIsource}{Certaines images proviennent du site freepics.com}


\newcommand{\UPSTItitreEnTete}{Automatisme pour la robotique}
\newcommand{\UPSTItitre}{Configuration du réseau de la baie du robot}
\newcommand{\UPSTInumero}{2}

\documentVersion{E}
\newcommand{\UPSTInumeroVersion}{1}
\begin{document}
\section{Préparation}
\subsection{Révisions}
\begin{UPSTIpreparation}[4][Cahier des charges à rédiger][Révisions - \SI{10}{min}]
    \UPSTIquestion{Ecrire un cahier des charges qu'un de vos camarade devra implémenter en TP.}

    Ce cahier des charges doit utiliser au minimum une entrée analogique, une entrée logique, une LED et une sortie console.
    \vspace{5cm}
\end{UPSTIpreparation}

\subsection{Compteurs et timers}


La première page de la documentation, indiquant les principaux périphériques et fonctionnalité comporte la phrase suivante :
\emph{Two 8-bit Timer/Counters with Separate Prescaler and Compare Mode}

\UPSTIquestion{Quel est l'intérêt que les prescalers et comparateurs soient séparés d'un timer à l'autre ?}

\section{Révisions}

\UPSTIboiteCentrale{Cahier des charges : Entrées et sorties logiques}{
    \begin{itemize}
        \item La diode D0 est allumée si le bouton poussoir BP0 est appuyé et le bouton poussoir BP1 est relâché.
        \item La diode D1 est allumée si la tension au bornes du potentiomètre est supèrieure à \SI{1.5}{V}
        \item La diode D2 est allumée si le bouton poussoir BP1 est appuyé pendant \SI{2}{s}
    \end{itemize}
}

\begin{UPSTIactivite}[4][Entrées/sorties Analogique et logique][Révisions - \SI{10}{min}]
    À partir du cahier des charges ci-dessus :
    \UPSTIetape{Implémenter le cahier des charges en utilisant les fonctions développées dans les TPs précédents.}
\end{UPSTIactivite}


\section{Time}

Le tableau suivant présente les registre des timers du ATmega2560 :

\begin{center}
    \begin{tabular}{|c|c|p{1.5cm}|p{1.5cm}|p{1.5cm}|p{1.5cm}|p{1.5cm}|p{1.5cm}|p{1.5cm}|p{1.5cm}|}
        \hline
        Timer & Registre & \multicolumn{8}{c}{Valeurs des bits}                     \\
        \hline
        \multirow[c]{2}{*}{1}     & TCCR1A   &  COM1A1 & COM1A0 & COM1B1 & COM1B0 & COM1C1     & COM1C0      & WGM11  & WGM10  \\\hline
        & & & & & & & & & \\\hline

    \end{tabular}
\end{center}

\end{document}
