\documentclass[TP, noCustomPackages]{UPSTI_Document}
\usepackage{listings}
\usepackage{lstautogobble}
\usepackage{hyperref}
\usepackage{fontawesome}
\usepackage{multirow}
\usepackage{multicol}
\usepackage{currfile}
\usepackage{pdfpages}
\usepackage{placeins}
%\usepackage{pdftricks}




% \begin{psinputs}
%     \usepackage{pstricks}
%     \usepackage{pstricks-add}
%     \usepackage{pst-eucl}
%     \usepackage{color}
%     \usepackage{pstcol}
%     \usepackage{pst-plot}
%     \usepackage{pst-tree}
%     \usepackage{pst-eps}
%     \usepackage{multido}
%     \usepackage{pst-node}
%     \usepackage{pst-eps}
% \end{psinputs}

\definecolor{vert}{rgb}{0.0, 0.5, 0.0}

\newlength{\iconeHeight}
\setlength{\iconeHeight}{15pt}

\newcommand{\UPSTIvariante}{2}


\newlist{todolist}{itemize}{2}
\setlist[todolist]{label=$\square$}
\newcommand{\UPSTIclasse}{S1}
\discipline[Autom - Robotique]{Automatisme pour la robotique}
\edef\CurrentFileDir{\currfiledir} 
\graphicspath{{\CurrentFileDir figures/}}



\newcommand{\basepath}{U:\textbackslashDocuments\textbackslashBUT\textbackslashGEII\textbackslashModulesS5\textbackslashAutomatisme_Pour_Robotique\textbackslash}
\newcommand{\subdir}[1]{\basepath{}#1}


%%% Définition du langage ST %%%
%New colors defined below
\definecolor{codegreen}{rgb}{0,0.6,0}
\definecolor{codegray}{rgb}{0.5,0.5,0.5}
\definecolor{codepurple}{rgb}{0.502,0.502,0.0}
\definecolor{backcolour}{rgb}{0.95,0.95,0.95}

\lstdefinelanguage{ST}
{
	morekeywords={
	case,of,if,then,end_if,end_case,super,function_block,extends,var,
	constant, byte,,end_var,var_input, real,bool,var_output,
	dint,udint,word,dword,array, of,uint,not,adr, program, for, end_for, while, do, end_while, repeat, end_repeat, until, to, by, else, elsif, var_in_out,interface,method,property
	},
	otherkeywords={
		:, :=, <>,;,\,.,\[,\],\^,1,2,3,4,5,6,7,8,9,0,TRUE, FALSE, \{attribute,  \'hide\'\}
	},
	keywords=[1]{
		case,of,if,then,end_if,end_case,super,function_block,extends,var,
		constant, byte,,end_var,var_input, real,bool,var_output,
		dint,udint,word,dword,array, of,uint,not,adr, :, :=, <>,;,\,.,\[,\],\^,program, for, end_for, while, do, end_while, repeat, end_repeat, until, to, by, else, elsif, var_in_out,interface,method,property
	},
	keywordstyle=[1]\color{blue},
	keywords=[2]{
		1,2,3,4,5,6,7,8,9,0, TRUE, FALSE
	},
	keywordstyle=[2]\color{codepurple},
	keywords=[3]{
		\{attribute,  \'hide\'\}
	},
	keywordstyle=[3]\color{codegray},
	sensitive=false,
	morecomment=[l]{//}, 
	morecomment=[s]{(*}{*)},
	morestring=[b]{"},
	morestring=[b]{'}
}

\lstset{
	language={ST},
	backgroundcolor=\color{backcolour},
	commentstyle=\color{codegreen}\textit,
	keywordstyle=\color{blue},
	numberstyle=\tiny\color{codegray},
	stringstyle=\color{codepurple},
	basicstyle=\ttfamily\scriptsize,
	breakatwhitespace=false,         
	breaklines=true,                 
	captionpos=b,                    
	keepspaces=true,                 
	numbers=left,                    
	numbersep=5pt,                  
	showspaces=false,                
	showstringspaces=false,
	showtabs=false,                  
	tabsize=2
}
%\lstMakeShortInline[columns=fixed]~
\newcommand{\UPSTItitreEnTete}{Informatique industrielle}
\newcommand{\UPSTItitre}{Conversions Analogique-Numérique}


\documentVersion{E}
\newcommand{\UPSTInumeroVersion}{1}
\begin{document}
\section{Révisions}

\UPSTIboiteCentrale{Cahier des charges : Entrées et sorties logiques}{
    \begin{itemize}
        \item La diode D0 est allumée si le bouton poussoir BP0 est appuyé et le bouton poussoir BP1 est relâché.
        \item La diode D1 est allumée si le bouton poussoir BP0 est relâché et le bouton poussoir BP1 est appuyé.
        \item La diode D2 est allumée si le bouton poussoir BP0 et le bouton poussoir BP1 sont appuyés.
        \item La diode D3 est allumée si le bouton poussoir BP0 et le bouton poussoir BP1 sont relâchés.
    \end{itemize}
}

\begin{UPSTIactivite}[4][Comportement combinatoire][Révisions - \SI{10}{min}]
    À partir du cahier des charges ci-dessus : 
    \UPSTIquestion{Donner l'expression logique de chacune des diodes.}
    \UPSTIetape{Implémenter le cahier des charges en utilisant les fonctions développées dans le TP 1.}
\end{UPSTIactivite}


\section{La conversion Analogique Numérique}

Ce TP a pour objectif de mettre en œuvre un CAN (Convertisseur Analogique Numérique) pour obtenir une valeur numérique image d'une grandeur analogique.
Les fonctions que nous développerons dans ce TP pourront être utilisées dans les futurs TPs ou dans vos projets. A ce titre, il est important pour chacune d'entre elles de respecter les bonnes pratiques citées ci-dessous. 

\begin{UPSTIinfor}{Bonnes pratiques à respecter pour chaque fonction}
    Pour chaque fonction développée, vous devrez documenter chacune de vos fonctions en utilisant la syntaxe vue au TP 1. 
\begin{todolist}
    \item respecter les conventions de nommage,
    \item avoir un nom explicite
    \item documenter la fonction : 
    \begin{todolist}
        \item en précisant son auteur et la date de création,
        \item en précisant son rôle,
        \item en précisant les paramètres d'entrée et de sortie,
    \end{todolist}
    \item tester la fonction : 
    \begin{todolist}
        \item pour un fonctionnement normal
        \item pour des cas limites
    \end{todolist}
\end{todolist}

\end{UPSTIinfor}

\begin{UPSTIactivite}[3][Création d'une bibliothèque][Préparation - \SI{3}{min}]
    Nous allons, tout d'abord, créer un dossier dans notre bibliothèque personnelle qui contiendra les fonctions liées au CAN.
    \UPSTIetape{Créer un dossier nommé \texttt{TP2\_CAN} dans le répertoire \texttt{lib/}}.
    \UPSTIetape{Créer deux fichiers \texttt{can.h} et \texttt{can.cpp} dans le dossier \texttt{lib/TP2\_CAN}}.
    \UPSTIetape{Dans le fichier \texttt{can.h}, ajouter la ou les ligne(s) permettant d'éviter les inclusions multiples.} 
\end{UPSTIactivite}

\begin{UPSTIactivite}[][Initialisation du CAN][\SI{10}{min}]
    \UPSTIquestion{Quelles sont les étapes nécessaires pour initialiser le CAN ?}
    \UPSTIetape{Définir une fonction permettant l'initialisation du CAN.}
    
    \emph{Rappel : le prototype devra se trouver dans le fichier \texttt{can.h} et le corps de la fonction dans le fichier \texttt{can.cpp}.}
    \UPSTIquestion{Dans quelle partie du programme (\emph{setup()} ou \emph{loop()}) devra-t-on appeler cette fonction ?}

    \emph{Cette fonction ne faisant rien d'autre qu'une initialisation, elle ne pourra pas être testée seule. Nous la testerons donc lors de l'implémentation des autres fonctions.}
\end{UPSTIactivite}


\begin{UPSTIactivite}[][Lecture du CAN sur 8 bits][\SI{15}{min}]
    \UPSTIquestion{Quelles sont les étapes nécessaires pour lire une valeur analogique ?}
    \UPSTIetape{Définir une fonction bloquante permettant la lecture d'une valeur analogique sur 8 bits.}
    \UPSTIquestion{Compléter la checklist suivante pour cette fonction : }
    \begin{todolist}
        \item Valeur pour une tension nulle : \dotfill
        \item Valeur pour une tension maximale : \dotfill
        \item Valeur à mi-course : \dotfill
    \end{todolist}
\end{UPSTIactivite}

\begin{UPSTIactivite}[][Test de la lecture du CAN sur 8 bits][\SI{5}{min}]
    \UPSTIetape{Écrire un programme permettant de tester la fonction de lecture du CAN en envoyant la valeur lue sur le port série.}
    \UPSTIquestion{Vérifier les points de la checklist suivante puis faire vérifier par l'enseignant. Si nécessaire, corriger la fonction. }
    \begin{todolist}
        \item Les fonctions d'initialisation et de conversion respectent les bonnes pratiques données en préambule. 
        \item La fonction de lecture donne des valeurs cohérentes avec la position du potentiomètre (cf Activité précédente)
        \item Sans la fonction d'initialisation, la fonction de lecture ne fonctionne pas.
    \end{todolist}
    \UPSTIetape{Archiver le programme dans votre dossier \texttt{TP2\_CAN}.}
\end{UPSTIactivite}

\begin{UPSTIactivite}[][Mesure du temps de conversion][\SI{5}{min}]
    \UPSTIquestion{D'après la documentation, quelle est la durée de conversion d'une valeur analogique ?}
    \UPSTIetape{Dans votre programme, ajouter des lignes permettant de mesurer le temps de conversion de votre fonction et de l'afficher sur le port série.}
    \UPSTIquestion{Quelle est la durée de conversion mesurée ? Est-elle en accord avec la valeur attendue ?}
\end{UPSTIactivite}

\begin{UPSTIactivite}[3][Erreur si temps de conversion trop long][\SI{10}{min}]
    \UPSTIetape{Modifier votre fonction de lecture pour qu'elle renvoie la valeur \si{-1} si le temps de conversion est trop long.}
\end{UPSTIactivite}

\UPSTIboiteCentrale{Cahier des charges : led selon la valeur lue par le CAN}{
    \begin{itemize}
        \item Toutes les diodes sont éteintes si la tension est inférieure à \SI{1}{V}.
        \item La diode D1 est allumée si la tension est comprise entre \SI{1}{V} et \SI{2}{V}.
        \item La diode D2 est allumée si la tension est comprise entre \SI{2}{V} et \SI{3}{V}.
        \item La diode D3 est allumée si la tension est comprise entre \SI{3}{V} et \SI{4}{V}.
        \item La diode D4 est allumée si la tension est supérieure à \SI{4}{V}.
    \end{itemize}
}

\begin{UPSTIactivite}[][LED selon la valeur lue par le CAN][\SI{7}{min}]
    \UPSTIetape{Implémenter le cahier des charges ci-dessus en utilisant les fonctions développées.}
\end{UPSTIactivite}

\begin{UPSTIactivite}[][Lecture du CAN sur 10 bits][\SI{15}{min}]
    \UPSTIquestion{Quelles sont les étapes supplémentaires nécessaires pour lire le résultat sur 10 bits ?}
    \UPSTIetape{Définir une fonction bloquante permettant la lecture d'une valeur analogique sur 10 bits.}
    \UPSTIquestion{Compléter la checklist suivante pour cette fonction : }
    \begin{todolist}
        \item Valeur pour une tension nulle : \dotfill
        \item Valeur pour une tension maximale : \dotfill
        \item Valeur à mi-course : \dotfill
    \end{todolist}
    \UPSTIquestion{Vérifier les points de la checklist suivante puis faire vérifier par l'enseignant. Si nécessaire, corriger la fonction. }
    \begin{todolist}
        \item La fonction respecte les bonnes pratiques données en préambule. 
        \item La fonction de lecture donne des valeurs cohérentes avec la position du potentiomètre
    \end{todolist}
    \UPSTIetape{Archiver le programme dans votre dossier \texttt{TP2\_CAN}.}
\end{UPSTIactivite}

\begin{UPSTIactivite}[][Mesure du temps de conversion sur 10 bits][\SI{5}{min}]
    \UPSTIetape{Mesurer le temps de conversion et le comparer avec la conversion 8 bits}
\end{UPSTIactivite}


\section{Application : Utilisation du Joystick}
Le Joystick de la carte est possède une résistance différente pour chaque axe. Il est donc possible de mesurer la position du joystick en utilisant le CAN.

\UPSTIboiteCentrale{Cahier des chrges : Utilisation du Joystick}{
    \begin{itemize}
        \item La diode D1 est allumée si le joystick est en position basse.
        \item La diode D2 est allumée si le joystick est en position haute.
        \item La diode D3 est allumée si le joystick est en position gauche.
        \item La diode D4 est allumée si le joystick est en position droite.
    \end{itemize}
}

\begin{UPSTIactivite}[][Valeurs du Joystick][\SI{15}{min}]
    \UPSTIetape{Implémenter le cahier des charges ci-dessus.}
\end{UPSTIactivite}

\section{Création d'une classe}
\begin{UPSTIactivite}[][Création d'une classe][\SI{20}{min}]
    \UPSTIquestion{Qu'est-ce qu'une classe ?}
    \UPSTIquestion{Quels sont les avantages de l'utilisation des classes ?}
    \UPSTIetape{Créer une classe nommée \texttt{CConvAN10bits} dans le dossier \texttt{lib/TP2\_CAN}.}
    \UPSTIetape{Définir la fonction d'initialisation du CAN dans cette classe.}
    \UPSTIetape{Définir la fonction de lecture sur 10 bits du CAN dans cette classe.}
    \UPSTIetape{Définir le constructeur de cette classe. Il appellera la fonction d'initialisation du CAN.}
    \UPSTIetape{Tester la classe en utilisant un programme similaire à celui de l'activité précédente.}
\end{UPSTIactivite}

\begin{UPSTIactivite}[][Affichage de la direction du Joystick][\SI{15}{min}]
    Le Joystick peut prendre les 8 positions suivantes : 
    \begin{multicols}{4}
        \begin{itemize}
            \item Bas
            \item Bas-Gauche
            \item Gauche
            \item Haut-Gauche
            \item Haut
            \item Haut-Droite
            \item Droite
            \item Bas-Droite
        \end{itemize}
    \end{multicols}
   
    \UPSTIetape{Afficher la direction du Joystick sur le port série en utilisant la classe définie.}
\end{UPSTIactivite}

\end{document}
