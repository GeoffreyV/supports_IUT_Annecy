\element{verins}
{
  \begin{question}{classe-abstraite}
    Quel est l'intérêt de définir une classe abstraite pour les vérins~?
    \evaluationProfLignes{1}{2}
  \end{question}

  \begin{question}{missions}
    Etablir la liste des missions à traiter pour la classe \emph{CVerin}.
    \evaluationProfLignes{1}{2}
    \end{question}

    \begin{question}{etats}
        Etablir la liste des états possibles pour un vérin.
        \evaluationProfLignes{1}{2}
    \end{question}

    \begin{question}{electrique}
        Etablir la facette électrique d'un vérin simple effet.
        \evaluationProfLignes{1}{2}
    \end{question}

    \begin{question}{interface}
        Quelle propriétés devront être associées à l'interface \emph{ITF\_ObjVerin}~? Préciser le rôle de chacune et les accesseurs à conserver.
        \evaluationProfLignes{1}{4}
    \end{question}

    \begin{question}{derivation}
        Ecrire la ligne de code définissant la classe \emph{CVerinSimpleEffet}. Vous indiquerez les héritages et la (les) interface(s) qu'elle implémente
        \evaluationProfLignes{1}{1}
    \end{question}

    \begin{question}{contexte}
        A partir de sa facette électrique et de ses propriétés, établir le contexte du bloc fonctionnel \emph{CVerinSimpleEffet}.
        \evaluationProfLignes{1}{3}
    \end{question}

    \begin{question}{accesseurs}
        Donner le corps des accesseurs associés aux propriétés définies. 
        \evaluationProfLignes{1}{3}
    \end{question}

    \begin{question}{constructeur}
        Etablir le constructeur \emph{FB\_Init} de la classe \emph{CVerinSimpleEffet}.
        \evaluationProfLignes{1}{2}
    \end{question}

    \begin{question}{constructeur-etat}
        Etablir le constructeur d'état de la classe \emph{CVerinSimpleEffet}. Compléter, au besoin, le contexte du bloc fonctionnel. 
        \evaluationProfLignes{1}{4}
    \end{question}

    \begin{question}{missionIN}
        Etablir la méthode de prise en compte des missions pour la classe \emph{CVerinSimpleEffet}.
        \evaluationProfLignes{1}{4}
    \end{question}

    \begin{question}{actionneurs}
        Etablir la commande des actionneurs pour la classe {CVerinSimpleEffet}.
        \evaluationProfLignes{1}{4}
    \end{question}

    \begin{question}{schedule}
        Etablir le corps de la méthode \emph{Schedule} de la classe \emph{CVerinSimpleEffet}.
        \evaluationProfLignes{1}{2}
    \end{question}

    \begin{question}{main}
        Donner la déclaration et l'instanciation du bloc fonctionnel \emph{CVerinSimpleEffet} dans le programme principal. Penser à lier les entrées et sorties.
        \evaluationProfLignes{1}{3}
    \end{question}
}