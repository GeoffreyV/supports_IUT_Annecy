\section{Architecture centralisée ou modulaire et répartie}
\label{sec:architecture}
Un système automatisé centralisé consiste en une unique unité centrale qui gère l'ensemble des entrées et des sorties. Cette unité centrale est donc reliée à l'ensemble des capteurs et des actionneurs.
Dans le domaine qui nous intéresse, l'unité centrale est un automate programmable industriel (API) et pourrait gérer une série d'étapes d'une chaîne de production. 

\begin{minipage}{0.45\linewidth}
    \paragraph{Avantages}
    \begin{itemize}
        \item Un automate unique.
        \item Architecture de données simple.
    \end{itemize}
\end{minipage}%
%
\begin{minipage}{0.45\linewidth}
    \paragraph{Inconvénients}
    \begin{itemize}
        \item Nombre d'entrées/sorties important.
        \item Traitement d'automatisation complexe.
        \item Maintenance et évolutions difficiles.
    \end{itemize}
\end{minipage}

A l'inverse, un système automatisé modulaire et réparti consiste en plusieurs unités centrales qui gèrent chacune une partie des entrées et des sorties. Chaque unité centrale est reliée à un ensemble de capteurs et d'actionneurs. Un protocole de communication est mis en place entre les unités centrales. 

\begin{minipage}{0.45\linewidth}
    \paragraph{Avantages}
    \begin{itemize}
        \item Nombre d'entrées/sorties limité par unité centrale.
        \item Maintenance et évolutions propres à chaque unité.
    \end{itemize}
\end{minipage}%
%
\begin{minipage}{0.45\linewidth}
    \paragraph{Inconvénients}
    \begin{itemize}
        \item Architecture de données complexe.
        \begin{itemize}
            \item Orientation des données.
            \item Perte de la sémantique des données qu'il faut reconstruire (union, chevauchement, \dots).
        \end{itemize}
        \item Protocole de communication à mettre en place.
    \end{itemize}
\end{minipage}

%% TODO: Insérer des images des différentes architectures %%% 

\paragraph{Critères de choix}
Le choix entre une architecture centralisée ou modulaire ainsi que les matériels et logiciels à utiliser se feront en fonction de différents critères : 
\begin{itemize}
    \item Nombre d'entrées/sorties TOR.
    \item Nombre d'entrées/sorties analogiques.
    \item Communications à mettre en place entre les unités centrales (protocoles, nombres).
    \item Maintenance et évolution.
    \item Interconnexion avec le Cloud (4.0).
    \item Puissance de calcul nécessaire.
    \item Boucle de régulation intégrée ou non.
    \item Coût (matériel, chaîne de développement, \dots).
    \item Langages IEC 61131-3 supportés.
    \item Les extensions de ces langages pouvant conduire à une programmation orientée objet.
\end{itemize}


%%% TODO: Réorganiser cette partie %%%
%%% TODO: Ajouter une figure architecture modulaire avec bus %%%

\section{Le système d'exploitation}
L'automate industriel programmable embarque un système d'exploitation temps réel. Ce système d'exploitation est un logiciel qui permet de gérer les ressources matérielles de l'automate (processeur, mémoire, entrées/sorties, \dots) et de fournir des services aux applications (gestion des tâches, gestion des communications, \dots). 

Pour programmer efficacement et correctement un automate, il est nécessaire de comprendre, au moins sommairement, le fonctionnement du système d'exploitation.

\UPSTIinfo[Système temps réel]{
    Un système d'exploitation temps réel garantit un temps de réponse maximum à toute requête ou événement matériel. 

    Ce type de système d'exploitation est utilisé dans les systèmes embarqués, les automates programmables, les systèmes de contrôle-commande, \dots En résumé, dans tous les domaines où la réactivité à un événement externe est primordiale.
}

\subsection{Propriétés du système d'exploitation d'un API}
Un système d'exploitation d'un API possède les propriétés suivantes : 
\begin{description}
    \item[Temps réel : ] par la mise en place d'un chien de garde (watchdog).
    \item[Interruptible : ] les tâches peuvent être interrompues par une tâche de \textbf{priorité supérieure} parce que l'événement qu'elle traite est \textbf{fugitif} ou d'\textbf{importance} pour l'application.
    \item[Mono/Multi-tâche : ] Plusieurs tâches peuvent être exécutées en parallèle. 
    \item[Accès facilité aux entrées/sorties].
    \item[Accès facilité à la mémoire : ] Permet de partager des données entre les tâches, de gérer les accès concurrents à la mémoire, \dots
    \item[Interaction avec le programme utilisateur]. 
\end{description}

Les événements sont capturés.
