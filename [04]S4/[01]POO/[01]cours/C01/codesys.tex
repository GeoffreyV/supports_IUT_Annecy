\section{Environnement de programmation : CoDeSys}
    Dans ce module, nous programmerons sous l'environnement de développement CoDeSys. 
    \subsection{Présentation}
    CoDeSys est un environnement de développement intégré (IDE) permettant de programmer des automates programmables industriels (API). Il est basé sur la norme IEC 61131-3. Il permet de programmer des automates de différentes marques (Beckhoff, Wago, Schneider, \dots). Il est disponible sous Windows et sous Linux. Il est gratuit pour une utilisation non commerciale. 
    \subsection{Langages}
    CoDeSys permet de programmer des automates en utilisant les langages suivants : 
    \begin{itemize}
        \item Instruction List (IL).
        \item Ladder Diagram (LD).
        \item Function Block Diagram (FBD).
        \item Structured Text (ST).
        \item Sequential Function Chart (SFC).
    \end{itemize}
    \subsection{Architecture d'un programme}
    % Un programme CoDeSys est composé de plusieurs fichiers :
    % \begin{itemize}
    %     \item Un fichier de configuration du projet (.pro).
    %     \item Un fichier de configuration de l'automate (.hwf).
    %     \item Un fichier de configuration du réseau (.net).
    %     \item Un fichier de configuration de la communication (.opc).
    %     \item Un fichier de configuration de la visualisation (.visu).
    %     \item Un fichier de configuration de la cible (.xml).
    %     \item Un fichier de configuration de la bibliothèque (.lib).
    % \end{itemize}

    CoDeSys permet de programmer en Monotâche et en Multitâche :  
    \begin{itemize}
        \item 15 tâches périodiques au maximum.
        \item Priorité des taches de 1 à 15, de la plus prioritaire à la moins prioritaire.
        \item Temps de cycle compris entre \SI{100}{\micro\second} et \SI{10}{\second}.
    \end{itemize}

    \subsection{Unité de Programme -- Program Organisation Unit (POU)}
    Une unité de programme est un ensemble d'instructions qui peuvent être appelées depuis un autre programme. Il existe 3 types de POU :

    \begin{description}
        \item[Programme (Program -- PRG) : ] Un programme consiste en une série d'instruction pouvant produire une ou plusieurs valeurs en sortie. Il peut être appelé depuis un autre programme ou depuis un bloc fonction.
        Toutes les valeurs des variables du programme restent inchangées entre deux appels, quelque soit sa provenance. 
        
        \item[Bloc fonction (Function Block -- FB) : ] Un bloc fonction est un ensemble d'instructions pouvant produire une ou plusieurs valeurs en sortie. Il peut être appelé depuis un autre programme ou depuis un bloc fonction. Un bloc fonction sera toujours appelé au travers d'une instance de bloc fonction, copie du bloc fonction. Par conséquent, les valeurs des variables du bloc fonction seront inchangées entre deux appels d'une même instance uniquement. Ce type de POU sera particulièrement adapté à la programmation orientée objet que nous développerons dans ce module. 
        \item[Fonction (Function -- FUN) : ] Une fonction ne peut produire qu'une seule valeur en sortie. Elle peut être appelée par n'importe quel POU et les valeurs des variables ne persistent pas d'un appel à l'autre. 
    \end{description}


    \UPSTIinfo[Structure d'un POU]{
    Un POU est divisé en 2 parties :
    \begin{enumerate}
        \item Déclaration des variables.
        \begin{itemize}
            \item Variables d'entrées/sorties : \textbf{VAR\_INPUT}, \textbf{VAR\_OUTPUT}, \textbf{VAR\_IN\_OUT}.
            \item Variables Locales : \textbf{VAR}, \textbf{CONSTANT}.
        \end{itemize}
        \item Implémentation du POU
    \end{enumerate}
    %%% TODO: Ajouter un exemple de POU %%%
    }

    \UPSTIinfo[METHOD et ACTION]{
        Au sein d'une POU, il est possible de définir des méthodes et des actions. 
        \begin{description}
            \item[ACTION] : Portion de code toujours accessible qui ne peut utiliser que le \textbf{contexte} de l'entité à laquelle elle est associée. 
            \begin{description}
                \item[Exemple : ] Une fonction RESET qui remet à zéro toutes les variables d'entrées/sorties d'un bloc fonction. 
            \end{description}
            \item[METHOD] :  Portion de code accessible qui utilise le contexte de l'entité à laquelle elle est associée mais qui peut également créer son propre contexte. On peut définir \textbf{l'accessibilité} d'une méthode (PRIVATE, PROTECTED, PUBLIC, INTERNAL, \dots).
        \end{description}
    }
\subsection{Les types de données}
\subsubsection{Les types de données natifs}
CodeSys propose un ensemble de types de données natifs. Ces types de données sont définis par la norme IEC 61131-3. 
Pour chaque type de données, le tableau suivant précise son empreinte mémoire ainsi que les règles de nommage (préfixe) que nous utiliserons dans ce module. 


\begin{table}[htbp]
    \centering
    \begin{tabular}{|c|c|c|}
        \hline
        \rowcolor{gray!30} \textbf{Type de données} & \textbf{Empreinte mémoire} & \textbf{Règles de nommage} \\
        \hline
        BOOL & 1 bit & x \\
        \hline
        BYTE & 8 bits & by \\
        \hline
        SINT & 8 bits & si \\
        \hline
        USINT & 8 bits & usi \\
        \hline
        WORD & 16 bits & w \\
        \hline
        INT & 16 bits & i \\
        \hline
        UINT & 16 bits & ui \\
        \hline
        DWORD & 32 bits & dw \\
        \hline
        DINT & 32 bits & d \\
        \hline
        UDINT & 32 bits & udi \\
        \hline
        REAL & 32 bits & r \\
        \hline
        LWORD & 64 bits & lw \\
        \hline
        LINT & 64 bits & li \\
        \hline
        ULINT & 64 bits & uli \\
        \hline
        LREAL & 64 bits & lr \\
        \hline
        STRING & Variable & s \\
        \hline
    \end{tabular}
    \caption{Empreinte mémoire et règles de nommage des types de données natifs}
    \label{tab:types_donnees}
\end{table}

\subsubsection{Les types de données de datation}
\begin{table}[htbp]
    \centering
    \begin{tabular}{|c|c|c|}
        \hline
        \rowcolor{gray!30} \textbf{Type de données} & \textbf{Empreinte mémoire} & \textbf{Règles de nommage} \\
        \hline
        TIME (T\#) & 32 bits & tim \\
        \hline
        TIME\_OF\_DAY (TOD\#) & 32 bits & tod \\
        \hline
        DATE (D\#) & 32 bits & date \\
        \hline
        DATE\_AND\_TIME (DT\#) & 32 bits & dt \\
        \hline
    \end{tabular}
    \caption{Empreinte mémoire et règles de nommage des types de données de datation}
\end{table}

\subsubsection{Les tableaux}
Sous l'environnement CoDeSys, il est possible de déclarer des tableaux de variables. Ils peuvent être de dimension 1, 2 ou 3. Aussi, et contrairement au langage C, il est possible de définir le premier et le dernier indice. \textbf{Ainsi, le premier indice d'un tableau n'est pas forcément 0.} 

\UPSTIinfo[Déclaration d'un tableau]{La déclaration d'un tableau se fait à l'aide de la syntaxe suivante : \\~<nom> : ARRAY [<ll1>..<ul2>|,<ll2> <ul2>|,<ll3>..<ul3>] OF <type>;~. \\Avec ~<lli>~ la borne inférieure et ~<uli>~ la borne supérieure de la dimension i. }

Notre convention de nommage veut que l'on ajoute le prefix ~a<dim><type>~ à la déclaration d'un tableau. Avec ~<dim>~ le nombre de dimensions et ~<type>~ le préfixe correspondant à celui des éléments du tableau (Cf Table \ref{tab:types_donnees}).

Par exemple, pour définir un tableau ~tab~ de taille 10 de type ~WORD~ et un tableau ~tab2~ de taille 20x10 de type ~INT~, on écrira :
\begin{lstlisting}[language=ST]
VAR 
    a1wtab : ARRAY [0..9] OF WORD;
    a2itab2 : ARRAY [1..20, 1..10] OF INT;
END_VAR
\end{lstlisting}

Il est possible de récupérer le premier et le dernier indice d'un tableau à l'aide des fonctions ~LOWER_BOUND(<array_name>,<dim>)~ et ~UPPER_BOUND(<array_name>,<dim>)~, respectivement.

\UPSTIinfo[Tableau à longueur variable]{
    Lorsqu'un tableau est donné en ~VAR_IN_OUT~ d'une fonction, d'un bloc fonction ou d'une méthode, il est possible de ne pas préciser la taille du tableau par la syntaxe ~<nom> : ARRAY [*|, *|, *] OF <type>~. Dans ce cas, le tableau est dit à longueur variable. Il est possible de récupérer la taille du tableau à l'aide de la fonction ~SIZEOF(<array_name>)~.

    Par exemple, pour définir un tableau ~tab~ de dimension 1 et de taille variable de type ~WORD~, on écrira :\\~a1wtab : ARRAY [*] OF WORD;~
}

\subsubsection{Les structures}
\begin{UPSTIinfor}{Définition d'une structure}
    La définition d'une structure se fait à l'aide de la syntaxe suivante : 
    \begin{lstlisting}[language=ST]
TYPE <struct_name> :
    STRUCT
        <nom1> : <type1>;
        <nom2> : <type2>;
        ...
        <nomi> : <typei>;
    END_STRUCT
END_TYPE\end{lstlisting}

    Avec ~<nomi>~ le nom de la variable et ~<typei>~ le type de la variable. 
\end{UPSTIinfor}
Sous l'environnement CoDeSys, il est possible de déclarer des structures de variables. 
Notre convention de nommage veut que l'on ajoute le prefix ~T~ au nom d'une variable de type structure et ~m_<data_type><member_name>~ au nom d'un membre de la structure.

Par exemple, pour définir une structure ~TVerin~ contenant deux membres ~m_xSorti~ et ~m_xRENTRE~ de type ~BOOL~, on écrira : 
\begin{lstlisting}[language=ST]
TYPE TVerin :
    STRUCT
        m_xSorti : BOOL;
        m_xRENTRE : BOOL;
    END_STRUCT
END_TYPE\end{lstlisting}

Dans une structure uniquement, il est possible de déclarer des membres de type ~BIT~ afin de ne réserver qu'un seul bit mémoire (contrairement au type BOOL qui réserve 1 octet mémoire). Cela peut permettre d'assigner un nom à différents bits d'un registre, par exemple. 

\paragraph{Accès aux membres d'une structure : }
Il est possible d'accéder aux membres d'une structure à l'aide de la syntaxe ~<struct_name>.<member_name>~. Par exemple, pour accéder au membre ~m_xSorti~ de la structure ~enrVerin1~ de type ~TVerin~, on écrira ~enrVerin1.m_xSorti~.


\paragraph{Héritage : }
Le mot clé ~EXTENDS~ permet de définir une structure héritant d'une autre structure. La structure fille hérite de tous les membres de la structure mère auxquels elle peut ajouter ses propres membres. La syntaxe de cet héritage simple est alors : 

\begin{lstlisting}[language=ST]
TYPE <struct_fille> EXTENDS <struct_mere> :
    STRUCT
        <nom_n+1> : <type_n+1>;
        <nom_n+2> : <type_n+2>;
        ...
    END_STRUCT
END_TYPE\end{lstlisting}

\subsubsection{Les énumérations}
Une énumération est un type permettant de nommer des constantes entières. Elle augmente la lisibilité du code et permet de le rendre plus robuste. On peut par exemple définir une énumération ~ETAT~ contenant les constantes ~ETAT\_OUVERT~, ~ETAT\_FERME~, ~ETAT\_EN\_COURS~ et ~ETAT\_ERREUR~. Chaque ETAT correspond à une valeur entière que le développeur ne connait pas forcément. 

Par défaut, le premier élément de l'énumération vaut 0 et chaque élément suivant vaut l'élément précédent + 1. Il est possible de modifier cette valeur en utilisant l'opérateur ~|:=~. Par défaut, le type des valeurs de chaque élément est ~INT~. 

\begin{UPSTIinfor}{Définition d'une énumération}
    La définition d'une énumération se fait à l'aide de la syntaxe suivante : 
    \begin{lstlisting}[language=ST]
TYPE <enum_name> :
    (
        <nom1> |:= <valeur1>,
        <nom2> |:= <valeur2>,
        ...
        <nomi> |:= <valeuri>
    )|<base_data_type>|:= <valeur_par_defaut>;
END_TYPE\end{lstlisting}
\end{UPSTIinfor}

Par exemple, pour définir une énumération ~ETAT~ contenant les constantes ~ETAT\_OUVERT~, ~ETAT\_FERME~, ~ETAT\_EN\_COURS~ et ~ETAT\_ERREUR~, on écrira :
\begin{lstlisting}[language=ST]
TYPE ETAT :
    (
        ETAT_OUVERT,
        ETAT_FERME,
        ETAT_EN_COURS,
        ETAT_ERREUR
    ):= ETAT_OUVERT;
END_TYPE\end{lstlisting}

Voici alors un exemple de déclaration, d'initialisation et d'utilisation d'une variable de type ~ETAT~ :
\begin{lstlisting}[language=ST]
VAR
    etat : ETAT;
END_VAR
etat := ETAT_FERME;
IF etat = ETAT_FERME THEN
    (* Faire quelque chose *)
END_IF\end{lstlisting}

\UPSTIattention[Noms de constante identiques]{
    Si deux types énumération contiennent des constantes de même nom, il est alors impératif de préciser le type d'énumération. Par exemple, si on définit une énumération ~ETAT~ contenant les constantes ~ETAT\_OUVERT~, ~ETAT\_FERME~, ~ETAT\_EN\_COURS~ et ~ETAT\_ERREUR~ et une énumération ~ETAT\_VERIN~ contenant les constantes ~ETAT\_OUVERT~, ~ETAT\_FERME~, ~ETAT\_EN\_COURS~ et ~ETAT\_ERREUR~, il faudra alors écrire ~ETAT.ETAT\_OUVERT~ et ~ETAT\_VERIN.ETAT\_OUVERT~ pour différencier les deux constantes.
}