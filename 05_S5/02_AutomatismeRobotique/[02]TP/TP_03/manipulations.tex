
\begin{UPSTIinfor}{Principe de la programmation robot}
    La mise en oeuvre du robot se fait par l'intermédiaire d'une queue (FIFO - First In First Out) de commande de mouvement. Les commandes seront donc effectuées par le robot dans l'ordre reçu et chacune d'entre elle sera retirée de la queue une fois exécutée. 
    
    Pour rappel : 
    \begin{itemize}
        \item L'exécution séquentielle des commandes de mouvement dans la pile débute dès que le bras est mis sous puissance (\texttt{MC\_GroupPower}).
        \item Il est possible d'ajouter des commandes de mouvement à tout moment 
        \begin{itemize}
            \item Mouvement en cours
            \item Robot à l'arrêt 
            \item Bras sous puissance
            \item Bras hors puissance
            \item ...
        \end{itemize}
        \item Une commande de mouvement sort de la pile dès que sa trajectoire associée est totalement achevée.
        \item Pour vider la FIFO de toutes les commandes de mouvement empilées, on dispose du bloc fonctionnel \texttt{MC\_GroupStop}.
        \item L'arrêt immédiat du robot durant l'exécution d'une trajectoire peut se faire 
        \begin{itemize}
            \item via le bouton d'arrêt d'urgence (E-Stop)
            \item par un changement de mode de marche sur la baie (WMS9)
            \item à l'aide des blocs fonctionnels \texttt{MC\_GroupInterrupt} et \texttt{MC\_GroupStop}.
        \end{itemize}
        \item Suite à un arrêt obtenu à l'aide du bloc fonctionnel \texttt{MC\_GroupInterrupt}, il est possible : 
        \begin{itemize}
            \item de reprendre les mouvements là où ils ont été stoppés grâce au bloc fonctionnel \texttt{MC\_GroupContinue}
            \item de les supprimer en vidant la pile (\texttt{MC\_GroupStop} vu précédemment).
        \end{itemize} 
        \item Le contenu de la pile des commandes de mouvement est conservé même si le bras est hors puissance. Une mise sous puissance du bras conduira éventuellement à une phase dite de « connexion » faite à petite vitesse pour rejoindre la trajectoire interrompue puis à la reprise des mouvements toujours présents dans la pile.
    \end{itemize}
    
    \end{UPSTIinfor}
    
    
    