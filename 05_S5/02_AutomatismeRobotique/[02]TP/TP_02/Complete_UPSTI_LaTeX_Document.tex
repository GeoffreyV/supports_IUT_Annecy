
\documentclass{UPSTI_Document}

\usepackage{UPSTI_Pedagogique}
\usepackage{UPSTI_SI}
\usepackage{hyperref}

\title{Mise en œuvre d’UniVALplc sous Ethernet/IP}
\author{IUT d'ANNECY / TETRAS}
\date{Année 2023-2024}

\begin{document}

\maketitle

\section{Introduction}
Les différentes structures de commande disponibles pour l'intégration de robots dans les systèmes de production 
sont variées et dépendent des compétences de l'intégrateur ainsi que de l'environnement d'application. Les 
contrôleurs modernes offrent des interfaces digitales et analogiques, et supportent divers protocoles de 
communication, tels que CanOpen, Profibus DP, et Ethernet industriel (EtherCAT, Ethernet/IP, PowerLink, Profinet).

\section{Objectifs}
\begin{itemize}
    \item Comprendre les principes de base de la solution UniVALplc de Stäubli
    \item Configurer le service IO-Scanning sous Ethernet/IP
    \item Accéder à la baie CS9 et aux E/S déportées du contrôleur M221
    \item Mettre en œuvre des blocs fonctionnels de la bibliothèque uniVALplc
\end{itemize}

\section{Préambule}
Ce TP vise à intégrer un automate programmable pour commander un robot dans une cellule de production, en utilisant la solution UniVALplc de Stäubli. 
Le contrôleur de robot, via son serveur uniVALplc, accepte des commandes d’un client API, comme le M340, à travers des blocs fonctionnels 
spécifiques.

\section{Matériel}
Le matériel requis pour ce TP comprend :
\begin{itemize}
    \item \textbf{Robot Stäubli SCARA TS240}: Robot collaboratif 4 axes avec contrôleur CS9 et mode sélecteur WMS9.
    \item \textbf{Convoyeur TS2plus}: Convoyeur Bosch RexRoth pour transporter les palettes avec produits.
    \item \textbf{Contrôleur PFC200 WAGO}: Pour la gestion du convoyeur et accessible via Ethernet/IP.
    \item \textbf{Automate Schneider M340}: Utilisé pour la communication avec la baie CS9 et le contrôleur M221.
    \item \textbf{Contrôleur M221}: Fournit les informations de supervision via Ethernet/IP.
\end{itemize}

\section{Préparation}
Avant de commencer, assurez-vous de :
\begin{enumerate}
    \item Importer les fichiers EDS nécessaires dans Control Expert
    \item Configurer la bibliothèque de types uniVALplc dans le logiciel
    \item S'assurer que les connexions réseau sont correctement établies et sécurisées
\end{enumerate}

\section{Configuration du Réseau Ethernet/IP}
\subsection{Cartographie de l'espace mémoire}
Les équipements CS9 et M221 sont configurés sur le réseau NOC\_ETHIP\_ROBOT avec les paramètres suivants :
\begin{table}[h]
    \centering
    \begin{tabular}{|l|c|c|}
        \hline
        Équipement & Entrée (T -> O) & Sortie (O -> T) \\
        \hline
        CS9 & 148 octets & 124 octets \\
        M221 & 4 octets & 40 octets \\
        \hline
    \end{tabular}
    \caption{Cartographie mémoire pour le réseau Ethernet/IP}
\end{table}

\subsection{Déclaration des objets mémoire}
Déclarez les objets `fromRobotScara` et `toRobotScara` pour représenter les données produites et consommées de la baie CS9, et les objets 
`fromHmiScara` et `toHmiScara` pour le contrôleur M221.

\section{Manipulations}
\begin{UPSTIactivite}
    \subsection{Configuration des adresses IP}
    Configurez les adresses IP pour chaque équipement selon le plan d'adressage.

    \subsection{Mise en œuvre des Blocs Fonctionnels}
    Utilisez les blocs fonctionnels \texttt{VAL\_ReadAxesGroup} et \texttt{VAL\_WriteAxesGroup} pour lire et écrire les données du robot.

    \subsection{Interface HMI}
    Utilisez les blocs \texttt{FB\_ReadHmi} et \texttt{FB\_WriteHmi} pour interagir avec l'interface opérateur et superviser les voyants.

    \subsection{Système de Préhension}
    Configurez le système de préhension via les sorties rapides TOR disponibles sur le connecteur J212 de la baie CS9.
\end{UPSTIactivite}

\section{Annexes}
\subsection{Blocs Fonctionnels}
Documentation des blocs fonctionnels \texttt{VAL\_ReadAxesGroup} et \texttt{VAL\_WriteAxesGroup}, qui permettent l'interaction avec le robot via 
le serveur UniVALplc. Ces blocs utilisent l'objet \texttt{T\_StaeubliRobot} pour représenter l'état du robot.

\subsection{Configurations réseau}
\begin{itemize}
    \item \textbf{Fichier EDS pour CS9}: Staeubli-CS9-uniVALplc-s4.6.eds
    \item \textbf{Fichier EDS pour M221}: M221-Scara.eds (modifié à partir de M221\_EDS\_Model.eds)
\end{itemize}

\end{document}
