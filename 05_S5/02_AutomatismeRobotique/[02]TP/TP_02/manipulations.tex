
\section{Partie TP - Manipulations}

\UPSTIattention{Pour ne pas avoir à recommencer en cas de problème, sauvegarder régulièrement le projet.}

\begin{UPSTIVerification}[Pré-requis]
    \UPSTIetape{Vérifier le fonctionnement des manipulations du TP précédent}
\end{UPSTIVerification}

\begin{UPSTIConfiguration}[Options du projet]
    \UPSTIetape{Dans \texttt{Outils->Options du projet}, configurer les options suivantes : }
    \begin{itemize}
        \item Autoriser le chevauchement d'adresses : 
        \begin{itemize}
            \item \texttt{Général} ->Gestion des messages de génération : le chevauchement d’adresses ne génère
            aucun message.
        \end{itemize}
        \item Autorisation des tableaux dynamiques : 
        \begin{itemize}
            \item \texttt{Variables} -> \texttt{Autoriser les tableaux dynamiques (\textbf{ANY\_ARRAY\_XXX})} : cocher la case.
            \end{itemize}
    \end{itemize}
\end{UPSTIConfiguration}



\subsection{Configuration du réseau du Robot}

Dans le TP précédent, nous avons configuré le premier coupleur \texttt{BMX NOC 0401.2} pour le réseau du guichet (\texttt{172.16.180.0/24}). Nous allons maintenant configurer le second coupleur pour le réseau \texttt{NOC\_ETHIP\_ROBOT (192.168.0.0/24)}.

\begin{UPSTIConfiguration}[Ajout des équipements sur le réseau \texttt{NOC\_ETHIP\_ROBOT}]
    \UPSTIetape{Vérifier la configuration de l'adresse IP}
    \begin{itemize}
        \item Ouvrir le \texttt{Navigateur de DTM} (\texttt{Outils->Navigateur de DTM} puis double-clic sur le réseau).
        \item Rubrique \texttt{TCP/IP} : vérifier que l'adresse IP est bien \texttt{192.168.0.1/24}.
        \end{itemize}
    \UPSTIetape{Ajout des équipements (pour chaque équipement : \textbf{Clique droit -> Ajouter})}
    \begin{itemize}
        \item Ajouter la baie CS9 
        \begin{description}
            \item [Type :] \texttt{uniVALplc v4.6 - CS9 Adapter}
            \item [Gestion des noms de \texttt{DTMs} :] cs9Scara
        \end{description}
        \item Ajouter le contrôleur M221
        \begin{description}
            \item [Type :] \texttt{TM221\_for\_Scara}
            \item [Gestion des noms de \texttt{DTMs} :] m221Scara
        \end{description}
    \end{itemize}
    \textit{Après avoir configuré les interfaces IP des coupleurs \texttt{BMX NOC 0401.2}, il est essentiel de respecter l'ordre d'ajout des équipements sur le réseau \texttt{NOC\_ETHIP\_ROBOT}, afin de respecter le plan d'adressage mémoire défini précédemment. }
\end{UPSTIConfiguration}
\pagebreak
\begin{UPSTIConfiguration}[Configuration des adresses des équipements]
    \UPSTIetape{Configurer les adresses des équipements}
    \begin{enumerate}
        \item Double-cliquer sur le coupleur réseau \texttt{NOC\_ETHIP\_ROBOT}.
        \item Pour chacun des appareils ajoutés, configurer les adresses IP suivantes (\textit{Onglet Paramétrage de l'adresse}) :
        \begin{itemize}
            \item \texttt{[003] cs9Scara} : \texttt{192.168.0.10}
            \item \texttt{[004] m221Scara} : \texttt{192.168.0.11}
        \end{itemize}
    \end{enumerate}
\end{UPSTIConfiguration}

\begin{UPSTIConfiguration}[Configuration des équipements]
    \UPSTIetape{Configurer le cs9Scara}
    \begin{itemize}
        \item Sur le \texttt{Navigateur de DTM}, double-cliquer sur le \texttt{cs9Scara}.
        \item Rubrique \texttt{Exclusive Owner} : Paramétrer la périodicité de la tâche Mast. 
    \end{itemize}
    \UPSTIetape{Configurer le m221Scara}
    \begin{itemize}
        \item Sur le \texttt{Navigateur de DTM}, double-cliquer sur le \texttt{m221Scara}.
        \item Supprimer le service \texttt{Write To 150} et le remplacer par un service \texttt{Read From 100 / Write To 150}. 
        \item Vérifier que ce nouveau service offre 4 octets en entrées et 40 octets en sortie, avec une lecture/écriture périodique de \SI{30}{ms}
    \end{itemize}
\end{UPSTIConfiguration}

\begin{UPSTIVerification}[Vérification de l'espace mémoire]
    \UPSTIetape{Sauvegarder, Vérifier et Compiler votre application}
    \UPSTIetape{Dans l'éditeur de données : Vérifier les zones mémoires allouées}
    \begin{itemize}[label=$\square$]
        \item \texttt{cs9Scara\_IN : T\_cs9Scara\_IN AT \%MW80}
        \item \texttt{cs9Scara\_OUT : T\_cs9Scara\_OUT AT \%MW208}
        \item \texttt{m221Scara\_IN : T\_m221Scara\_IN AT \%MW154}
        \item \texttt{m221Scara\_OUT : T\_m221Scara\_OUT AT \%MW270}
    \end{itemize}
\end{UPSTIVerification}

