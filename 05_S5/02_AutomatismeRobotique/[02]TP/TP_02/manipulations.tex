
\section{Partie TP - Manipulations}

\UPSTIattention{Pour ne pas avoir à recommencer en cas de problème, sauvegarder régulièrement le projet.}

\begin{UPSTIManipulation}[Pré-requis]
    \UPSTIetape{Vérifier le fonctionnement des manipulations du TP précédent}
\end{UPSTIManipulation}

\subsection{Configurations préalables}
\subsubsection{Installation de la bibliothèque UniVALplc}
La mise en œuvre d'UniVALplc repose sur l'importation des fichiers \texttt{.eds} et sur l'ajout de la bibliothèque client UniVALplc dans \textit{Control Expert}. Ces opérations nécessitent des droits d'administrateur sur les stations de travail.
Ainsi, ces manipulations ont été faites en amont du TP. 

\begin{UPSTIinfor}{Importation des Fichiers EDS et de la Bibliothèque de Types}
    \textit{Ces manipulations ont été faites en amont du TP. Cet encart est à titre informatif.}
    \begin{enumerate}
        \item Exécuter le programme de mise à jour de la bibliothèque de types sous \textit{Control Expert} en mode administrateur.
        \item Importer le fichier \texttt{FAMILY.DCS} pour ajouter la bibliothèque UniVALplc à l'environnement de développement.
    \end{enumerate}
    \textit{Les fichiers et bibliothèques nécessaires sont disponible dans le dossier \texttt{Scara/uniVALplc/}}
\end{UPSTIinfor}

\subsubsection{Options du projet}
\begin{UPSTIConfiguration}[Options du projet]
    \UPSTIetape{Dans \texttt{Outils->Options du projet}, configurer les options suivantes : }
    \begin{itemize}
        \item Autoriser le chevauchement d'adresses : 
        \begin{itemize}
            \item \texttt{Général} ->Gestion des messages de génération : le chevauchement d’adresses ne génère
            aucun message.
        \end{itemize}
        \item Autorisation des tableaux dynamiques : 
        \begin{itemize}
            \item \texttt{Variables} -> \texttt{Autoriser les tableaux dynamiques (\textbf{ANY\_ARRAY\_XXX})} : cocher la case.
            \end{itemize}
    \end{itemize}
\end{UPSTIConfiguration}



\subsection{Configuration du réseau du Robot}

Dans le TP précédent, nous avons configuré le premier coupleur \texttt{BMX NOC 0401.2} pour le réseau du guichet (\texttt{172.16.180.0/24}). Nous allons maintenant configurer le second coupleur pour le réseau \texttt{NOC\_ETHIP\_ROBOT (192.168.0.0/24)}.

\begin{UPSTIManipulation}[Ajout des équipements sur le réseau \texttt{NOC\_ETHIP\_ROBOT}]
    \UPSTIetape{Vérifier la configuration de l'adresse IP}
    \begin{itemize}
        \item Ouvrir le \texttt{Navigateur de DTM} (\texttt{Outils->Navigateur de DTM} puis double-clic sur le réseau).
        \item Rubrique \texttt{TCP/IP} : vérifier que l'adresse IP est bien \texttt{192.168.0.1/24}.
        \end{itemize}
    \UPSTIetape{Ajout des équipements (pour chaque équipement : \textbf{Clique droit -> Ajouter})}
    \begin{itemize}
        \item Ajouter la baie CS9 
        \begin{description}
            \item [Type :] \texttt{uniVALplc v4.6 - CS9 Adapter}
            \item [Gestion des noms de \texttt{DTMs} :] cs9Scara
        \end{description}
        \item Ajouter le contrôleur M221
        \begin{description}
            \item [Type :] \texttt{TM221\_for\_Scara}
            \item [Gestion des noms de \texttt{DTMs} :] m221Scara
        \end{description}
    \end{itemize}
\end{UPSTIManipulation}

Après avoir configuré les interfaces IP des coupleurs \texttt{BMX NOC 0401.2}, il est essentiel de respecter l'ordre d'ajout des équipements sur le réseau \texttt{NOC\_ETHIP\_ROBOT}, afin de valider le plan d'adressage mémoire.

Les équipements suivants doivent être ajoutés dans cet ordre :
\begin{itemize}
    \item Baie CS9 (\texttt{uniVALplc v4.6 - CS9 Adapter}).
    \item Contrôleur M221 (\texttt{TM221\_for\_Scara}).
\end{itemize}

Vérifiez que les adresses IP attribuées correspondent bien au plan d'adressage défini précédemment (figure 1).

\subsection{Communication via Ethernet/IP}

\subsubsection{Configuration des Sections de Programme}

Deux sections de programme sont à créer dans la tâche principale (\textit{Mast}) :
\begin{itemize}
    \item \textbf{ReadEthIpRobot} : cette section lira les données du robot à l'aide du bloc fonctionnel \texttt{VAL\_ReadAxesGroup} et l'état de l'interface HMI à l'aide du bloc \texttt{FB\_ReadHmi}.
    \item \textbf{WriteEthIpRobot} : cette section fournira les informations de supervision à l'HMI et écrira les commandes de mouvement du robot avec le bloc \texttt{VAL\_WriteAxesGroup}.
\end{itemize}

Le contrôle de l'exécution des blocs fonctionnels sera assuré par les bits de santé (\texttt{HEALTH\_BITS\_IN[0]}) provenant de la baie CS9 et du contrôleur M221.

\subsubsection{Compilation et Tests}

\begin{enumerate}
    \item Compiler et transférer l'application sur l'API.
    \item Mettre en \textit{RUN} l'automate et vérifier la communication via la table d'animation, en observant l'état de santé de la baie CS9 et du contrôleur M221.
    \item Sur l'interface HMI, visualiser l'état du robot et agir sur les éléments de commande (boutons, voyants, etc.).
\end{enumerate}
