\subsection{Méthodes}
\label{subsec:methodes}
L'objet \lstinline{METHOD} est un outil proposé par CodeSys pour la programmation orientée objet. Il permet de définir des méthodes (fonctions) propres à un contexte avec un niveau d'accessibilité paramétrable (publique, privée, protégée ou interne).

Les avantages d'utiliser des méthodes au sein d'une classe sont les suivants : 
\begin{itemize}
    \item La modularité des blocs fonctionnels est favorisée
    \begin{itemize}
        \item On pourra décomposer le corps d'un bloc fonctionnel en plusieurs appels de méthodes
    \end{itemize}
    \item La lisibilité du code est améliorée
    \item l'intégrité des données est favorisée par l'utilisation judicieuse d'attributs d'accessibilité
\end{itemize}

Comme pour une fonction, on peut définir :
\begin{itemize}
    \item Des variables d'Entrées, de sorties ou d'entrées-sorties
    \item Des variables locales à la méthode
          \begin{itemize}
              \item La durée de vie de ces variables locales est alors limitée à l'exécution de la méthode
          \end{itemize}
\end{itemize}

Par ailleurs, \textbf{une méthode a accès au contexte de l'instance à laquelle elle est associée}. Cela signifie qu'elle a accès aux variables et méthodes de l'objet dont elle dépend.

Enfin, CodeSys autorise la déclaration de \lstinline{METHOD}s dans une interface. Elle devra alors être implémentée dans les classes qui implémentent cette interface.

\UPSTIaRetenir{Les méthodes des classes que nous définirons seront des méthodes associées au bloc fonction concerné.

Leur ligne de déclaration est la suivante : \lstinline{METHOD <Attribut> <Method_Name> | : <return_data_type>}}

\begin{UPSTIidee}{Les méthodes de la classe CStep}
        \begin{minipage}[t]{.6\linewidth}
            Les méthodes de la classe \emph{CStep} seront définies dans le bloc fonctionnel \emph{FB\_CStep}. Elles seront donc des méthodes associées à ce bloc fonctionnel.
        \paragraph{Méthodes}
        \begin{itemize}
            \item \textbf{MInit} : Méthode appelée à l'initialisation du programme
            \item \textbf{MActivate} : active l'étape
            \item \textbf{MDeactivate} : désactive l'étape
        \end{itemize}
    \end{minipage}%
    \begin{minipage}[t]{.35\linewidth}
        \begin{lstlisting}
METHOD PUBLIC MInit : BOOL
    VAR_INPUT
    END_VAR
    VAR_OUTPUT
    END_VAR
    VAR
    END_VAR
    // Code de la méthode
    // ...
    // Fin du code de la méthode
END_METHOD
        \end{lstlisting}
    \end{minipage}
\end{UPSTIidee}