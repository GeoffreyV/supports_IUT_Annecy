\section{Gmma : Graphe des modes de marche et d'arrêt}

On propose de mettre en oeuvre une commande hiérarchisée suivant un GEMMA. 
Dans ce contexte : toute erreur de réseau (NOC\_ETHIP\_CONVOYEUR ou NOC\_ETHIP\_ROBOT) doit être vue comme une panne de la partie opérative et placer le système dans l'état D1.
Le système reste dans cet état tant que la communication n'est pas rétablie.

Pour chacun des mode, ce sujet précise les blocs fonctionnels qui pourront être utilisés. 

\subsection{Mode D1 : Arrêt d'Urgence}
Ce mode est le premier mode actif lorsque l'automate M340 passe en RUN.
Dans ce mode : 
\begin{itemize}
    \item Le bras est mis hors puissance (MC\_GroupPower) dès que cela est possible, c'est à dire : 
    \begin{itemize}
        \item La bibliothèque \texttt{UnivalPlc} est initialisée (T\_StaubliRobot.Status.Initialized) 
        \item La baie CS9 est prête à recevoir des commandes (T\_StaubliRobot.Status.Online)
    \end{itemize}
    \item Le système de préhension est maintenu dans son état.
    \item Les bloqueurs du Poste Guichet doivent en interdire l'accès.
\end{itemize}

Pendant ce mode, la colonne lumineuse de l'IHM (\texttt{hmiScrara.m\_Outputs.m\_qToR.m\_nqx*Column}) doit être :
\begin{itemize}
    \item Rouge si le bras a pu être mis hors puissance,
    \item Rouge clignotant à la fréquence de 10 Hz sinon.
\end{itemize}

\subsection{Mode D2 : Diagnostic/Traitement de la défaillance}
Ce mode a pour objectif de donner une configuration saine du système. C'est à dire : 
\begin{enumerate}
    \item Acquitter les défauts présents (MC\_GroupReset),
    \item Vider la file des commandes de mouvements (MC\_GroupStop),
    \item Autoriser de nouvelles commandes de mouvement (MC\_GroupContinue),
    \item Mettre le bras sous puissance (MC\_GroupPower).
\end{enumerate}

Pendant ce mode, le guichet est interdit d'accès. La colonne lumineuse de l'IHM est au \texttt{jaune} fixe.

\subsection{Mode A5 : Remise En Route après Défaillance}

Ce mode sera détaillé dans un prochain TP. Il permettra d'évacuer une pièce restée saisie.
Pour le moment, on considère que l'opérateur autorise la dépose par le biais du pupitre opérateur (bouton bleu « autorisation de dépose » -- \texttt{J212-2/J212-7} de la CS9) et signale que toutes les aires de traitement sont vides (bouton « cuves vides » de l'IHM).

Pendant ce mode, le guichet est interdit d'accès. La colonne lumineuse de l'IHM est au \texttt{jaune} clignotant à la fréquence de 1 Hz.

\subsection{Mode A6 : Mise En Conditions Initiales}

On souhaite positionner le robot au-dessus du guichet, prêt à saisir un gobelet sans entraver la circulation des palettes vides ou pleines.
Les coordonnées cartésiennes dans le repère « world » respectant ces contraintes sont : (X=90, Y=-410, Z=150, RX=-180, RY = 0, RZ =0).

Si la position courante n'est pas celle voulue, on adopte la stratégie suivante :
\begin{enumerate}
    \item Monter l'axe J3 en position 150 mm sans modifier la position courante des autres axes.
    \item Diriger le robot au-dessus du guichet pour terminer le positionnement voulu par une commande de mouvement de type \textbf{articulaire}.
\end{enumerate}
Durant toute la durée de ce mode : 
\begin{itemize}
    \item Si un arrêt est demandé via le pupitre opérateur : 
    \begin{itemize}
        \item Immobiliser le robot tout en maintenant le bras sous puissance.
        \item Reprendre la course dès que l'arrêt n'est plus actif.
        \item Ce mode est signalé par l'allumage clignotant rapide (\SI{10}{ms}) du voyant rouge associé au bouton « arrêt demandé » ainsi que de la colonne lumineuse.
    \end{itemize}
    \item Le guichet est interdit d'accès. Les deux bloqueurs (BE, BP) sont levés.
\end{itemize}

\subsection{Mode A1 : Arrêt en conditions initiales}
Le bras du robot sera immobile mais sous puissance au-dessus du guichet passant. La colonne sera \textit{Cyan} afin d'indiquer à l'opérateur que la cellule est prête pour traiter des pièces en mode automatique s'il le souhaite.

\subsection{Dans tous les modes}
\begin{itemize}
    \item Le voyant blanc \texttt{Bras sous puissance} de l'IHM est allumé si et seulement si le bras est sous puissance. 
\end{itemize}
