\section{Manipulations}
\label{sec:manipulations}
\subsection{Préparation du GMMA}
\label{subsec:preparation_gmma}

Dans notre commande hiérarchisée, un programme principal \texttt{GM\_Gmma} aura pour rôle de gérer les différents modes de marche du robot. C'est celui-ci qui autorisera ou non chaque mode de marche. Pour cela, on utilisera une section par mode de marche. Chacune des sections sera conditionnée à une variable booléenne commandée par le programme principal.

Le Gmma sera mis en oeuvre à l'aide de trois sections exécutés dans l'ordre suivant :
\begin{enumerate}
    \item \texttt{GM\_Gmma\_Init}
    \begin{itemize}
        \item Lors du premier cycle automate : 
        \begin{itemize}
            \item Initialisation du GMMA lors du premier cycle automate (à l'aide du bloc fonctionnel \texttt{InitChart}),
            \item Configuration de la période du bit de vie du robot à \SI{150}{ms} au sein de l'objet \texttt{staubliRobotScara},
        \end{itemize}
        \item Mise en défaut du GMMA en cas d'erreur de communication.
    \end{itemize}
    \item \texttt{GM\_Gmma\_Main}
    \begin{itemize}
        \item Gestion des différents modes de marche du robot
    \end{itemize}
    \item \texttt{GM\_Gmma\_Post}
    \begin{itemize}
        \item Gestion des variables booléennes de chaque mode de marche,
        \item Fournir l'état du GMMA à l'IHM pour un affichage sur le 7 segments.
    \end{itemize}
\end{enumerate}

\begin{UPSTImanipulation}[Préparation de la structure du programme]
    Dans un premier temps, nous allons créer le squelette de notre programme avec les différentes sections qui seront complétées tout au long de ce TP.
    \UPSTIetape{Créer les trois sections de gestion du GMMA.}
    \UPSTIetape{Réserver une variable booléenne pour chaque mode de marche du robot.}
    \UPSTIetape{Créer une section par mode de marche. Chaque section sera conditionnée à la variable booléenne correspondante. \texttt{Clic droit -> Caractéristiques -> Condition}}
\end{UPSTImanipulation}

\begin{UPSTImanipulation}[Programmation du GMMA]
    \UPSTIetape{Dans la section \texttt{GM\_Gmma\_Init}}
    \begin{itemize}
        \item Initiliser le GMMA à l'aide du bloc fonctionnel \texttt{InitChart},
        \item Configurer la période du bit de vie du robot à \SI{150}{ms} au sein de l'objet \texttt{staubliRobotScara},
    \end{itemize}
    \UPSTIetape{En cas d'erreur de communication avec le robot, mettre en défaut le GMMA.}
\end{UPSTImanipulation}

