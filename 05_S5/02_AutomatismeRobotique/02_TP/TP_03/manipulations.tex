\section{Manipulations}
\label{sec:manipulations}
\subsection{Squelette du programme}
\label{subsect:squelette_du_programme}

Dans notre commande hiérarchisée, un programme principal \texttt{GM\_Gmma} aura pour rôle de gérer les différents modes de marche du robot. C'est celui-ci qui autorisera ou non chaque mode de marche. Pour cela, on utilisera une section par mode de marche. Chacune des sections sera conditionnée à une variable booléenne commandée par le programme principal.

Le Gmma sera mis en oeuvre à l'aide de trois sections exécutés dans l'ordre suivant :
\begin{enumerate}
    \item \texttt{GM\_Gmma\_Init}
          \begin{itemize}
              \item Lors du premier cycle automate :
                    \begin{itemize}
                        \item Initialisation du GMMA lors du premier cycle automate (à l'aide du bloc fonctionnel \texttt{InitChart}),
                        \item Configuration de la période du bit de vie du robot à \SI{150}{ms} au sein de l'objet \texttt{staubliRobotScara},
                    \end{itemize}
              \item Mise en défaut du GMMA en cas d'erreur de communication.
          \end{itemize}
    \item \texttt{GM\_Gmma\_Main}
          \begin{itemize}
              \item Gestion des différents modes de marche du robot
          \end{itemize}
    \item \texttt{GM\_Gmma\_Post}
          \begin{itemize}
              \item Gestion des variables booléennes de chaque mode de marche,
              \item Fournir l'état du GMMA à l'IHM (\texttt{hmiScara.m\_Outputs.m\_Screen.m\_intStateGmma}).
          \end{itemize}
\end{enumerate}

\begin{UPSTIManipulation}[Préparation de la structure du programme]
    Dans un premier temps, nous allons créer le squelette de notre programme avec les différentes sections qui seront complétées tout au long de ce TP.
    \UPSTIetape{Créer les trois sections de gestion du GMMA.}
    \UPSTIetape{Créer une section par mode de marche. Chaque section sera conditionnée à une variable booléenne correspondante. \texttt{Clic droit -> Caractéristiques -> Condition}}
\end{UPSTIManipulation}

\begin{UPSTIManipulation}[Programmation du GMMA]
    \UPSTIetape{Compléter la section \texttt{GM\_Gmma\_Init}}
    \UPSTIetape{Compléter la section \texttt{GM\_Gmma\_Main}, les conditions pourront être écrites ulterieurement.}
    \UPSTIetape{Compléter la section \texttt{GM\_Gmma\_Post}}
\end{UPSTIManipulation}

\subsection{Programmation des différents modes de marche}
Pour chacun des modes de marches, on propose la démarche suivante :

\begin{UPSTIinfor}{Démarche}
    \begin{itemize}
        \item Mode combinatoire :
              \begin{enumerate}
                  \item Réserver les éventuelles instances de blocs fonctionnels nécessaires
                  \item Compléter la section correspondante en ST ou LADDER
              \end{enumerate}
        \item Mode séquentiel :
                \begin{enumerate}
                    \item Etablir le programme SFC de ce mode 
                    \item Créer les actions P1 pour configurer les mouvements à réaliser dans les étapes correspondantes
                    \item Compléter la section d'appel des blocs fonctionnels associés 
                    \item Compléter la section \textit{post} pour exécuter les différentes actions. 
                \end{enumerate}
    \end{itemize}
\end{UPSTIinfor}

\begin{UPSTIManipulation}[Modes D1 et D2]
    \UPSTIetape{Compléter le mode D1}
    \UPSTIetape{Compléter le mode D2}
\end{UPSTIManipulation}

\begin{UPSTIVerification}[Mode D1 et D2]
    \UPSTIetape{Vérifier, compiler et téléverser le programme}
    \UPSTIetape{\textbf{Faire vérifier par l'enseignant}}
    \begin{itemize}[label=$\square$]
        \item Voyant blanc allumé sur le pendantif de la CS9
        \item Voyant blanc \texttt{Bras sous puissance} allumé sur l'IHM
        \item Guichet interdit d'accès
        \item Colonne lumineuse de l'IHM au jaune fixe pour le mode D2
    \end{itemize}
\end{UPSTIVerification}

\begin{UPSTIManipulation}[Modes A5]
    \UPSTIetape{Compléter le mode A5}
\end{UPSTIManipulation}

\begin{UPSTIVerification}[Mode A5]
    \UPSTIetape{Vérifier, compiler et téléverser le programme}
    \UPSTIetape{\textbf{Faire vérifier par l'enseignant}}
    \begin{itemize}[label=$\square$]
        \item Guichet interdit d'accès
        \item Colonne lumineuse de l'IHM au jaune clignotant à la fréquence de 1 Hz
    \end{itemize}
\end{UPSTIVerification}

\begin{UPSTIManipulation}[Modes A6]
    \UPSTIetape{Compléter le mode A6 en suivant la démarche proposée plus haut. }
\end{UPSTIManipulation}

\begin{UPSTIVerification}[Mode A6]
    \UPSTIetape{\textbf{Faire vérifier par l'enseignant avant tout versement sur l'automate}}
    \UPSTIetape{Compiler, téléverser et vérifier le fonctionnement}
\end{UPSTIVerification}

\begin{UPSTIManipulation}[Mode A1]
    \UPSTIetape{Compléter le mode A1 puis faire valider par l'enseignant}
\end{UPSTIManipulation}