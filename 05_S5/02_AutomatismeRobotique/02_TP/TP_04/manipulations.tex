\section{Partie TP}
On respectera les bonnes pratiques de programmation définies dans les TPs précédents, à savoir \textit{(liste non exhaustive)} :
\begin{itemize}
    \item Utilisation des macros-étapes pour simplifier la lecture du programme.
    \item Utilisation de variables structurées pour accéder aux paramètres des blocs fonctionnels.
    \item Appels des blocs fonctionnels dans une section dédiée, associée au mode concerné.
\end{itemize}

\begin{UPSTIinfor}{Conventions de nommage}
    \begin{itemize}
        \item Les macros-étapes commencent par le nom du mode suivi de \texttt{\_MS}, d'un numéro puis de l'action qu'elle effectue. Exemple : \texttt{F1\_MS1\_SaisieGobelet}.
        \item Une action sur front montant (\texttt{P1}) ou sur front descendant (\texttt{P0}) commence par le nom du mode suivi, puis de l'éventuelle macro à laquelle elle appartient, de de \texttt{\_P0} ou \texttt{\_P1}puis l'action qu'elle effectue. Exemple : \texttt{F1\_MS1\_AP1\_ConfigurationMvtsLineaires}.
        \item Les variables structurées associées à un bloc fonctionnel commencent par \texttt{s\_} suivi du mode dans lequel elles sont utilisées, suivi du nom du bloc fonctionnel, puis du nom de la variable structurée. Exemple : \texttt{s\_F1\_VAL\_ShiftPoint}.
        \item Les instances de blocs fonctionnels commencent par le nom du mode, suivi d'un nom explicite du bloc fonctionnel. Exemple : \texttt{F1\_ShiftPoint}.
    \end{itemize}
\end{UPSTIinfor}

\begin{UPSTIManipulation}[Implémentation]
    Dans notre cas, le mode \texttt{F1} sera donc composé de 3 sections : 
\begin{description}
    \item[F1\_ProductionNormale] : Contient le SFC principal du mode \texttt{F1}.
    \item[F1\_ProductionNormale\_Actions] : Contient les actions associées aux étapes du mode \texttt{F1}. 
    \item[F1\_ProductionNormale\_AppelsBF] : Contient les appels des blocs fonctionnels associés au mode \texttt{F1}.
\end{description}
    \UPSTIetape{Implémenter les transitions \texttt{T\_A1\_to\_F1} \texttt{T\_F1\_to\_A2} et \texttt{T\_A2\_to\_A1}.}
    \UPSTIetape{Modifier le SFC du mode \texttt{F1} pour qu'il corresponde à la préparation.}
    \UPSTIetape{Implémenter la macro-étape \texttt{F1\_MS1} en respectant les consignes données dans les préparations.}
    \UPSTIetape{Implémenter la macro-étape \texttt{F1\_MS2} en respectant les consignes données dans les préparations.}
    \UPSTIetape{Instancier les blocs fonctionnels et les variables structurées associées.}
    \UPSTIetape{Compléter la section d'appel des blocs fonctionnels associée au mode \texttt{F1}.}
    \UPSTIetape{Compléter la section \textit{post} associée au mode \texttt{F1}.}
    \UPSTIetape{Faire vérifier avant de tester}
\end{UPSTIManipulation}