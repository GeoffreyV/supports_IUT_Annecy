\section{Partie TD}
\subsection{Mode de marche Normal}
Pour simplifier l'élaboration du programme, on utilisera des macros-étapes : 
\begin{itemize}
    \item \texttt{F1\_MS1} : Saisie des gobelets.
    \item \texttt{F1\_MS2} : Dépose des gobelets.
\end{itemize}
\begin{UPSTIPreparation}[Grafcet du mode F1]
	\UPSTIquestion{Donner le grafcet du mode F1. On ne décrira pas, pour le moment, les étapes des macro-étapes \texttt{F1\_MS1} et \texttt{F1\_MS2}.}
	\vspace{17cm}
\end{UPSTIPreparation}



\subsection{Implémentation des macro-étapes}
On prendra comme position de référence la position initiale du robot (\texttt{X=90, Y=-410, Z=150, RX=-180, RY=0, RZ=0}). \textbf{Les autres positions seront calculées par rapport à cette position d'origine. Cette transformation sera effectuée à l'aide de blocs fonctionnels issus de la bibliothèque \texttt{uniVALplc}.}

\begin{UPSTIPreparation}[Blocs fonctionnels de transformation de coordonnées]
	\UPSTIquestion{Lire la documentation du type \texttt{T\_Trsf} et du bloc fonctionnel \texttt{VAL\_ShiftPoint}, puis donner les étapes pour pour transformer les coordonnées d'un point en utilsant ce bloc.}
	\vspace{4cm}
\end{UPSTIPreparation}


\begin{UPSTIPreparation}[Macro-étape \texttt{F1\_MS1}]
	\UPSTIquestion{Proposer un grafcet pour la macro-étape\texttt{F1\_MS1}.} 
    
    \textit{A côté de chaque étape, décrire les actions effectuées et les variables qui devront être modifiées.}

	\vspace{11cm}
	\textbf{Faire vérifier}
\end{UPSTIPreparation}
