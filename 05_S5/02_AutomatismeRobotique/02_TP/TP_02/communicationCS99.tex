
\subsection{Mise en oeuvre de la communication}
Dans cette section, nous allons créer les sous-programmes en charge de la communication avec la baie CS9. 
On propose, pour cela, d'ajouter deux sous-programmes à notre projet avec les noms et rôles suivants : 
\begin{enumerate}
    \item Une section \texttt{ReadEthIpRobot}, \textbf{première section à être exécutée}, elle aura pour rôle :
    \begin{enumerate}
        \item Lecture des données du robot via le bloc fonctionnel \texttt{VAL\_ReadAxesGroup}
        \item Lecture des informations depuis l'IHM via le bloc fonctionnel \texttt{BF\_ReadHmi}
    \end{enumerate} 
    \item Une section \texttt{WriteEthIpRobot}, \textbf{dernière section à être exécutée}, elle aura pour rôle :
    \begin{enumerate}[resume]
        \item Écriture des informations vers l'IHM via le bloc fonctionnel \texttt{BF\_WriteHmi}
        \item Écriture des données du robot via le bloc fonctionnel \texttt{VAL\_WriteAxesGroup}
    \end{enumerate}
\end{enumerate}


Pour cela, ce TP suivra les étapes suivantes : 
\begin{enumerate}
    \item Importation des blocs fonctionnels nécessaires (HMI et CS9) dans le projet
    \item Déclaration des variables nécessaires à la communication
    \item Création des sections \texttt{ReadEthIpRobot} et \texttt{WriteEthIpRobot}
    \begin{enumerate}
        \item Réservation des variables 
        \item Instanciation des blocs fonctionnels
        \item Appel des blocs de lecture et d'écriture dans les sections respectives. 
    \end{enumerate}
    \item Vérification du bon fonctionnement de la communication
\end{enumerate}

\subsubsection{Communication avec la baie CS9}

Les blocs fonctionnels \texttt{VAL\_ReadAxesGroup} et \texttt{VAL\_WriteAxesGroup} sont fournis par la bibliothèque \texttt{UnivalPlc}. Puisque l'ajout d'une bibliothèque de types nécessite des droits d'administrateur, cela a été fait en amont du TP.

\begin{UPSTIinfor}{Importation des Fichiers EDS et de la Bibliothèque de Types}
    \textit{Ces manipulations ont été faites en amont du TP. Cet encart est à titre informatif.}
    \begin{enumerate}
        \item Exécuter le programme de mise à jour de la bibliothèque de types sous \textit{Control Expert} en mode administrateur.
        \item Importer le fichier \texttt{FAMILY.DCS} pour ajouter la bibliothèque UniVALplc à l'environnement de développement.
    \end{enumerate}
    \textit{Les fichiers et bibliothèques nécessaires sont disponible dans le dossier \texttt{Scara/uniVALplc/}}
\end{UPSTIinfor}


\begin{UPSTIManipulation}[Déclaration des variables nécessaires à la communication]
    Afin de rassembler les données provenant du bloc de communication avec le robot, on propose de déclarer une structure de données \texttt{T\_CtrlStatusGroup}. 
    \lstinputlisting[language=st]{TCtrlStatusGroup.st}

    \UPSTIetape{Définir ce type de données dans le dossier \texttt{Types données dérivés} du projet.}
\end{UPSTIManipulation}
    

\begin{UPSTIManipulation}[Mise en oeuvre de la communication]
    \UPSTIetape{Créer les deux sous-programmes décrits en début de cette sous-section et les ordonner afin qu'ils soient exécutés dans l'ordre adéquate.}
    \UPSTIetape{Réserver les variables du tableau suivant. Pour certaines variables, l'adresse de lecture est à fixer selon la préparation (cf. \ref{sec:declarationObjets}).}
    \begin{center}
        \begin{tabular}{|l|l|l|l|}
            \hline
            Nom & Type & Adresse & Commentaire \\
            \hline
            ctrlStatusGroupRead & TCtrlStatusGroup &  & \textit{État de la lecture} \\
            ctrlStatusGroupWrite & TCtrlStatusGroup &  & \textit{État de l'écriture} \\
            fromRobotScara & T\_FromRobot & \textit{à fixer} & \textit{Données du robot} \\
            toRobotScara & T\_ToRobot & \textit{à fixer} & \textit{Commandes vers le robot} \\
            staeubliRobotScara & T\_StaeubliRobot &  & \textit{Structure de données du robot} \\
            \hline
        \end{tabular}
    \UPSTIetape{Instancier les blocs fonctionnels \texttt{VAL\_ReadAxesGroup}, \texttt{VAL\_WriteAxesGroup}, \texttt{BF\_ReadHmi} et \texttt{BF\_WriteHmi} dans le dossier \texttt{Instances FB dérivé}.}
    \UPSTIetape{Appeler les blocs fonctionnels dans les sections \texttt{ReadEthIpRobot} et \texttt{WriteEthIpRobot} en respectant l'ordre de lecture et d'écriture.}
    \begin{itemize}
        \item On conditionnera l'exécution des blocs au bon fonctionnement de la communication. Pour cela, prendre en compte le bit de santé \texttt{NOC\_ETHIP\_ROBOT\_IN.HEALTH\_BITS\_IN[0].0}.
    \end{itemize}
    \end{center}
\end{UPSTIManipulation}

\begin{UPSTIVerification}[Fonctionnement de la communication]
    \UPSTIetape{Créer une table d'animation nommée \texttt{scara} afin de visualiser les données suivantes :}
    \begin{multicols}{2}
        \begin{itemize}
            \item \texttt{NOC\_ETHIP\_ROBOT\_IN.HEALTH\_BITS\_IN[0].0}
            \item \texttt{ctrlStatusGroupRead}
            \item \texttt{ctrlStatusGroupWrite}
            \item \texttt{staeubliRobotScara}
        \end{itemize}
    \end{multicols}
    
    \UPSTIetape{Vérifier, compiler puis transférer le programme.}
    \UPSTIetape{Vérifier que les données échangées entre l'IHM et le robot sont correctes.}
    \begin{itemize}[label=$\square$]
        \item Bit de santé du coupleur 
        \item Absence d'erreur dans les variables \texttt{ctrlStatusGroupRead} et \texttt{ctrlStatusGroupWrite}
        \item Conformité des données du robot \texttt{staeubliRobotScara}
    \end{itemize}
    
\end{UPSTIVerification}

\subsubsection{Communication avec le contrôleur M221}

Les blocs fonctionnels \texttt{BF\_ReadHmi} et \texttt{BF\_WriteHmi} ont été créés par l'équipe enseignante et doivent être importés dans le projet.

\begin{UPSTIManipulation}[Importation des blocs fonctionnels HMI]
    \UPSTIetape{\textbf{Clique-droit->Importer} sur le dossier \texttt{Type FB dérivés} dans l'arborescence du projet.}
    \UPSTIetape{Sélectionner les fichiers \texttt{FB\_ReadHmi.xdb} et \texttt{FB\_WriteHmi.xdb} dans le dossier \path{U:\Documents\BUT\GEII\ModulesS5\Automatisme_Pour_Robotique\Scara}}
    \UPSTIetape{Si l'application vous le demande, \textbf{Garder tout} puis valider.}
\end{UPSTIManipulation}

\begin{UPSTIManipulation}[Déclaration des variables nécessaires à la communication]
    \UPSTIetape{En suivant les mêmes étapes que pour la communication avec la baie CS9, réaliser la communication avec le contrôleur M221.}
    \begin{itemize}
        \item L'appel du bloc \texttt{BF\_ReadHmi} se fera \textbf{après} le bloc \texttt{VAL\_ReadAxesGroup}.
        \item L'appel du bloc \texttt{BF\_WriteHmi} se fera \textbf{avant} le bloc \texttt{VAL\_WriteAxesGroup}.
        \item On utilisera le bit de santé \texttt{NOC\_ETHIP\_ROBOT\_IN.HEALTH\_BITS\_IN[0].1} ainsi que le précédent pour conditionner l'exécution des blocs.
    \end{itemize}
\end{UPSTIManipulation}

\begin{UPSTIVerification}[Communication avec l'IHM]
    \UPSTIetape{Compiler, transférer puis vérifier le bon fonctionnement.}
    \UPSTIetape{Faire valider par l'enseignant.}
\end{UPSTIVerification}

