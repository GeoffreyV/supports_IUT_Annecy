
\subsection{Mise en œuvre de la solution UniVALplc}

La mise en œuvre d'UniVALplc repose sur l'importation des fichiers \texttt{.eds} et sur l'ajout de la bibliothèque client UniVALplc dans \textit{Control Expert}. Ces opérations nécessitent des droits d'administrateur sur les stations de travail.

\subsubsection{Importation des Fichiers EDS et de la Bibliothèque de Types}

\begin{enumerate}
    \item Exécuter le programme de mise à jour de la bibliothèque de types sous \textit{Control Expert} en mode administrateur.
    \item Importer le fichier \texttt{FAMILY.DCS} pour ajouter la bibliothèque UniVALplc à l'environnement de développement.
\end{enumerate}

\subsection{Configuration des Réseaux}

Après avoir configuré les interfaces IP des coupleurs \texttt{BMX NOC 0401.2}, il est essentiel de respecter l'ordre d'ajout des équipements sur le réseau \texttt{NOC\_ETHIP\_ROBOT}, afin de valider le plan d'adressage mémoire.

Les équipements suivants doivent être ajoutés dans cet ordre :
\begin{itemize}
    \item Baie CS9 (\texttt{uniVALplc v4.6 - CS9 Adapter}).
    \item Contrôleur M221 (\texttt{TM221\_for\_Scara}).
\end{itemize}

Vérifiez que les adresses IP attribuées correspondent bien au plan d'adressage défini précédemment (figure 1).

\subsection{Communication via Ethernet/IP}

\subsubsection{Configuration des Sections de Programme}

Deux sections de programme sont à créer dans la tâche principale (\textit{Mast}) :
\begin{itemize}
    \item \textbf{ReadEthIpRobot} : cette section lira les données du robot à l'aide du bloc fonctionnel \texttt{VAL\_ReadAxesGroup} et l'état de l'interface HMI à l'aide du bloc \texttt{FB\_ReadHmi}.
    \item \textbf{WriteEthIpRobot} : cette section fournira les informations de supervision à l'HMI et écrira les commandes de mouvement du robot avec le bloc \texttt{VAL\_WriteAxesGroup}.
\end{itemize}

Le contrôle de l'exécution des blocs fonctionnels sera assuré par les bits de santé (\texttt{HEALTH\_BITS\_IN[0]}) provenant de la baie CS9 et du contrôleur M221.

\subsubsection{Compilation et Tests}

\begin{enumerate}
    \item Compiler et transférer l'application sur l'API.
    \item Mettre en \textit{RUN} l'automate et vérifier la communication via la table d'animation, en observant l'état de santé de la baie CS9 et du contrôleur M221.
    \item Sur l'interface HMI, visualiser l'état du robot et agir sur les éléments de commande (boutons, voyants, etc.).
\end{enumerate}
