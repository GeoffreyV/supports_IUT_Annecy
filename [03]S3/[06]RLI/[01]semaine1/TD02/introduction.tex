On s'intéresse dans ce TD à la rénovation de la gestion technique du siège d'Audiens situé à Vanves et construit en 2005. Le bâtiment propose 24000 m² de locaux répartis sur 6 niveaux.

Un extrait du CCTP (Cahier des Clauses Techniques Particulières) est donné en annexe. Il présente les exigences du client en terme de gestion technique du bâtiment. Certaines question y feront référence. 

\section{Architecture du résea}

\UPSTIquestion{Quels sont les réseaux proposés dans le CCTP, page 10, pour la GTB ?}
%\UPSTIquestion{Lequel de ces réseaux allez-vous utiliser pour la gestion technique de l'énergie électrique ?}
%\UPSTIquestion{Lequel de ces réseaux allez-vous utiliser pour le contrôle-commande de la partie CVC ?}
\UPSTIquestion{Lequel ou lesquels de ces réseaux allez-vous utiliser pour la supervision ?}

\section{Gestion de l'éclairage}
La plus part des informations concernant l'éclairage est donné en page 15 du CCTP.

\UPSTIquestion{Quel est le type de réseau généralement utilisé pour la gestion de l'éclairage ?}

Le document n'impose pas l'utilisation d'un bus pour l'éclairage.

\UPSTIquestion{Quels sont les équipements proposés pour la gestion de l'éclairage ?}
\UPSTIquestion{Quelles sont les fonctions demandées pour l'éclairage ?}
\UPSTIquestion{Quel serait, pour les fonctions demandées, l'intérêt d'utiliser un bus pour l'éclairage ?}
\UPSTIquestion{Quelles sont les fonctionnalités supplémentaires que vous pourriez proposer pour justifier une solution comprenant un bus ?}
\UPSTIquestion{Le document n'impose pas de niveau d'éclairage. En cherchant sur internet la réglementation d'éclaire de la RE2020, que proposez-vous comme niveaux d'éclairage minimum dans les bureaux, dans les couloirs, dans les parking ?}
3\UPSTIquestion{On recommande une puissance d’éclairage entre 6 et 12 W/m². Chaque étage ayant une surface de 4000 m² et chaque lampe faisant environ 100W, combien de luminaires seront nécessaires ?}
\UPSTIquestion{Combien de luminaires peut-on installer sur un même bus DALI ?} 

\UPSTIquestion{Un étage comprend environ 250 luminaires. Quelle solution proposez-vous ?}

Le bus DALI v2 permet d'y accoupler des capteurs multifonction : (luminosité, présence, réception télécommande) et des boutons poussoirs. Ces capteurs et boutons sont alimentés par le bus.

\UPSTIquestion{Donnez un schéma synoptique du réseau DALI d'un étage comprenant :}
\begin{itemize}
	

\item un automate WAGO avec une ou des bornes DALI,
\item le réseau électrique 230 Vac,
\item 5 branches de 50 ballasts DALI (ne pas dessiner les 250 ballasts!),
\item l'alimentation 18 Vdc DALI
\item 5 ensembles de 4 capteurs multifonctions

\end{itemize}

\section{Programmation}
Pour une salle de projection, on propose le cahier des charges suivant :

\UPSTIboiteCentrale{Cahier des charges : Salle de projection}{
    \begin{itemize}
        \item L'appui sur un bouton poussoir \textbf{ixBpLuminaire} permet d'allumer ou d'éteindre la salle. 
        \item Un appui long sur le bouton poussoir permet de faire varier l'intensité lumineuse de la salle.
        \begin{itemize}
			\item Alternativement, l'appui long fera varier l'intensité vers le haut ou vers le bas.
			\item L'intensité lumineuse varie de 0 à 100\% en 10 secondes.
			\item L'intensité lumineuse varie de 100 à 0\% en 5 secondes.
		\end{itemize}
    \end{itemize}}

\UPSTIquestion{A partir de l'annexe, expliquer le principe de fonctionnement du bloc \textit{FbDALI\_DimmSingleButton}}


