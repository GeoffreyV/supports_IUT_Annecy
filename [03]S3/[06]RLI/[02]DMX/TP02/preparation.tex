\begin{UPSTIactivite}[2][Le protocole RS 485][][][Préparation]
	\label{prepa:rs485}%
	\UPSTIquestion{En cherchant sur internet, expliquer le fonctionnement du protocole RS 485.}
    \begin{itemize} 
        \item Quelles tensions sont utilisées ? 
        \item Quelle est la vitesse de transmission ?
        \item Quelle est la longueur maximale du bus ?
        \item Quelle est la différence avec le protocole RS 232 ?
        \item Expliquer la notion de bit de parité et de bit de stop.
        \vspace{5cm}
        \end{itemize}

    \UPSTIquestion{Comparer ce bus avec le Bus DALI déjà étudié.}
    \vspace{5cm}

    \UPSTIquestion{Le protocole RS485 est-il compatible avec la mise en oeuvre d'une communication DMX ?}
    \vspace{1cm}
\end{UPSTIactivite}

Une trame DMX512 contient dans sa partie utile 512 octets, qui par leur position dans la trame définit la valeur de
commande des 512 canaux. La norme laisse le soin à chaque constructeur de choisir l’interprétation de ces valeurs
(comprises entre 0 et 255).
\begin{UPSTIactivite}[2][Vitesse du DMX][][][Préparation]
     \UPSTIquestion{Combien de trame par seconde peut-on envoyer via le réseau DMX ?}
     \vspace{2cm}
\end{UPSTIactivite}


Dans ce TP, on se propose d'utiliser un coupleur \textbf{750-652} pour mettre en oeuvre un commande DMX. Il jouera le rôle de maître DMX. 

On reliera deux ballasts RGBW (Red Green Blue White). Sur ces deux ballasts, on fixe le canal de référence à l'aide de boutons poussoirs \textit{CH+} et \textit{CH-}. Ce canal de référence correspond à la sortie \textbf{R}, puis les canaux suivants seront respectivement \textbf{G}, \textbf{B} puis \textbf{W}. 

Dans notre cas, on donnera comme canaux de référence respectivement \textbf{1 puis 7}. 

\begin{UPSTIactivite}[2][Architecture matérielle][][][Préparation]
    \UPSTIquestion{Expliquer ce que fait le module \textbf{750-652}.}\\
    \vspace{1cm}

    Les éléments DMX de notre maquette sont installés en \textit{Daisy Chain}.
    \UPSTIquestion{Expliquer ce que signifie \textit{Daisy Chain}.}
    \vspace{1cm}
    \UPSTIquestion{Dessiner l'architecture matérielle permettant de mettre en oeuvre deux bandeaux RGBW à l'aide d'un automate Wago et du protocole DMX. }
    \vspace{10cm}
    \UPSTIquestion{Préciser sur le schéma les numéros des canaux DMX.}
\end{UPSTIactivite}

\pagebreak
\subsection{Programmation}

\begin{UPSTIactivite}[2][DMX Master][][][Préparation]
    \UPSTIquestion{A l'aide de l'annexe page \pageref{appendix:DMXMaster}, expliquer le rôle du bloc \textit{FbDMX\_Master}.}
    \vspace{5cm}
    \UPSTIquestion{De quel type et à quoi sert la variable \textbf{abDMX\_Values}?}
    \vspace{3cm}
    \UPSTIquestion{Proposer une classe associée à ce bloc fonction.}
    \vspace{8cm}
\end{UPSTIactivite}
\pagebreak
\UPSTIboiteCentrale{Cahier des charges : Une couleur}{
    On souhaite allumer la colonne de gauche (Canal 1) d'une couleur de votre choix.
}
\begin{UPSTIactivite}[2][Une couleur][][][Préparation]
    
    \UPSTIquestion{Choisir une couleur mélangeant les trois couleurs primaires puis écrire son code HTML ainsi que les valeurs d'intensité des trois canaux au format \textit{byte}.}
    \vspace{1cm}

    \UPSTIquestion{Écrire un programme, en langage CFC (Blocs fonctions), permettant d'allumer la colonne de gauche de la couleur choisie.}
    \vspace{8cm}

\end{UPSTIactivite}
\UPSTIboiteCentrale{Cahier des charges : potentiomètre}{
    La couleur du bandeau doit maintenant changer lorsque la valeur du potentiomètre \textit{bValue} est supérieure à 100.
}

\begin{UPSTIactivite}[2][Potentiomètre][][][Préparation]
    \UPSTIquestion{En ajoutant une partie de texte structuré, modifier le programme précédent pour répondre au cahier des charges.}
    \vspace{4cm}
\end{UPSTIactivite}


\pagebreak
\UPSTIboiteCentrale{Cahier des charges : Succession de couleurs}{
    \label{CDC:succession}
    On souhaite effectuer une séquence de 10 couleurs différentes sur la colonne de droite (Canal 7).
    La séquence de couleur est définie dans le tableau suivant : 

    \begin{tabular}{|ccccccccccc|}
        \hline
    B & 16\#00 & 16\#00 & 16\#00 & 16\#00 & 16\#00 & 16\#08 & 16\#10 & 16\#20 & 16\#40 & 16\#80 \\\hline
    G & 16\#08 & 16\#10 & 16\#20 & 16\#40 & 16\#80 & 16\#80 & 16\#40 & 16\#20 & 16\#10 & 16\#08 \\\hline
    R & 16\#80 & 16\#40 & 16\#20 & 16\#10 & 16\#08 & 16\#00 & 16\#00 & 16\#00 & 16\#00 & 16\#00 \\\hline
    \end{tabular}
}



\begin{UPSTIactivite}[2][Succession de couleurs][][][Préparation]
    Pour ce cahier des charges, on propose d'utiliser le bloc fonction \textit{FbRGB\_CrossFadeSequence} (page~\pageref{appendix:RGBCross}) qui permet de faire varier la couleur d'un canal de manière progressive.\\
    \UPSTIquestion{Proposer une classe associée à ce bloc fonction. On utilisera un tableau pour stocker les 10 couleurs.}
    \vspace{4cm}

    \UPSTIquestion{Écrire un programme, en langage ST, permettant d'allumer la colonne de gauche de la couleur choisie.}
    \vspace{10cm}

\end{UPSTIactivite}
